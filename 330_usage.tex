Как уже было описано выше, проблема оценки релевантности имеет большое практическое значение, поскольку от качества данной оценки (иными словами,
верно выполненного ранжирования результатов поиска) напрямую зависит конкурентоспособность поисковой системы, ее применяющей. Отсюда вытекает
и основное практическое предназначение результатов, полученных в диссертационной работе --- а именно, использование предложенных нейросетевых
моделей, алгоритмов и методов оценки релевантности непосредственно в программном обеспечении семантических поисковых систем.

Научное же предназначение данных результатов состоит в возможности их использования при проведении дальнейших исследований по теме
проблемы оценки релевантности семантического поиска, поскольку было показано, что данный подход является сравнительно новым и может
представлять интерес к исследованию вопросов, смежных с обсуждаемыми в данной работе.