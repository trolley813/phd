\subsection{Создание базы поисковых запросов}
Для обучения генеративно-состязательной сети, которое является основной целью  создания такой базы, как правило,
необходима достаточно обширная база знаний. Это означает, что ручное написание запросов и фильтрация результатов
в данном исследовании является довольно неэффективным методом решения подзадачи, поскольку требует серьезных
трудозатрат, которые невозможно автоматизировать. Как следствие, это сокращает исследовательский потенциал автора
исследования, поскольку подобные задачи заняли бы довольно существенный объем времени.

В связи с вышесказанным, необходимо выявить и определить методы для автоматического создания подобной базы.
Следует отметить, что тестирование проектируемой программной системы будет выполняться также с использованием 
поисковых запросов (отличных от тех, на которых производилось обучение сети), поэтому автоматическая генерация
будет полезна и для данного этапа решения поставленной задачи.

Наиболее перспекивным для данной цели также представляется нейросетевой подход, тем более, что разработан ряд
классов нейронных сетей, решающих поставленную задачу и дающих при этом вполне приемлемые результаты.