\subsection{Создание базы поисковых запросов}
Для обучения генеративно-состязательной сети, которое является основной целью создания такой базы, как правило,
необходима достаточно обширная база знаний. Это означает, что ручное написание запросов и фильтрация результатов
в данном исследовании является довольно неэффективным методом решения подзадачи, поскольку требует серьезных
трудозатрат, которые невозможно автоматизировать. Как следствие, это сокращает исследовательский потенциал автора
исследования, поскольку подобные задачи заняли бы довольно существенный объем времени.

В связи с вышесказанным, необходимо выявить и определить методы для автоматического создания подобной базы.
Следует отметить, что тестирование проектируемой программной системы будет выполняться также с использованием 
поисковых запросов (отличных от тех, на которых производилось обучение сети), поэтому автоматическая генерация
будет полезна и для данного этапа решения поставленной задачи.

Наиболее перспективным для данной цели также представляется нейросетевой подход, тем более, что разработан ряд
классов нейронных сетей, решающих поставленную задачу и дающих при этом вполне приемлемые результаты.

Одним из таких классов является модель GPT-2 \cite{radford2019language}, равно как и ее более ранняя версия ---
GPT-1 \cite{Radford2018ImprovingLU} (по состоянию на 2021 год представлено и третье поколение данной модели ---
GPT-3 \cite{brown2020language}).  Аббревиатура GPT обозначает "<генеративный предварительно обученный трансформер">
(англ. generative pre-trained transformer). Трансформер --- это архитектура нейронных сетей глубокого обучения,
предложенная в 2017 году исследователями из Google Brain \cite{vaswani2017attention}. Трансформеры предназначены
прежде всего для решения задач обработки последовательностей, коими являются, в частности, тексты на естественном языке.

Структурно трансформер состоит из подсетей --- кодировщика и декодировщика. Кодировщик получает на вход векторизованную 
последовательность с информацией о позициях, декодировщик же получает на вход часть этой последовательности и результат
с выхода кодировщика. 
Кодировщик и декодировщик используют слоистую топологию. Каждый слой в кодировщике последовательно передает результат 
следующему слою в качестве его входа, слои же декодировщика последовательно передают результат следующему слою 
вместе с результатом кодировщика в качестве его входа.

Кодировщик состоит из так называемого механизма самовнимания (вход из предыдущего слоя) и нейронной сети
с прямой связью (вход из механизма самовнимания). Декодировщик также состоит из механизма самовнимания, механизма внимания 
к результатам кодирования (вход из механизма самовнимания и кодировщика) и нейронной сети с прямой связью (вход из механизма внимания). 

Результат внимания (attention) входного вектора $X$ к вектору $Y$ вычисляется по формуле
\begin{equation}
    \text{Attention}(Q, K, V) = \text{softmax}\left(\frac{QK^\mathrm{T}}{\sqrt{d_k}}\right)V
\end{equation} 

Здесь вектора $Q=W_Q X$, $K = W_K X$, $V = W_V Y$, где матрицы $W$ с индексами --- это весовые матрицы, которыми параметризован
механизм внимания: матрица весов запросов $W_Q$, весов ключей $W_K$ и весов значений $W_V$.

\subsection{Анализ конфигурации нейронных сетей для модели оценки релевантности}
Под конфигурацией нейронной сети здесь подразумевается не только топология (архитектура) нейросетевых моделей, но также и 
ее параметризация (а именно, выявление параметров, определяющих релевантность запроса, и их подача на нейронную сеть).

Анализ подобных конфигураций проводится как теоретическим, так и экспериментальным путем. К теоретическим методам анализа,
в частности, относятся:
\begin{itemize}
    \item подбор параметризации --- выявление параметров, определяющих релевантность запроса, путем исследования существующих 
          (как классических, так и нейросетевых) алгоритмов решения поставленной проблемы;
    \item исследование существующих топологий нейронных сетей, подбор таковой, наиболее адекватной для выбранной параметризации.
\end{itemize}
Экспериментальные методы включают в себя 
\begin{itemize}
    \item собственно обучение нейронных сетей, экспериментальное определение вычислительной сложности процесса обучения;
    \item сравнение результатов, полученных для различных топологий и параметризаций, как между собой, так и с существующими классическими алгоритмами.
\end{itemize}

Требования к нейросетевой архитектуре для решения проблемы оценки релевантности таковы:
\begin{enumerate}[1)]
    \item обучение нейронной сети должно производиться за разумное (не слишком большое) время на ПК пользовательского класса;
    \item пополнение поисковой базы новыми документами не должно требовать переобучения используемых нейронных сетей (за исключением отдельных 
          "<экстренных"> случаев, вероятность которых весьма низка);
    \item результаты, даваемые нейронной сетью (оценки релевантности для последующего ранжирования), должны быть, во всяком случае, не хуже 
          (то есть, с меньшей или равной вероятностью ошибки), чем таковые с использованием классических алгоритмов ранжирования.
\end{enumerate}

В работе рассматриваются преимущественно генеративно-состязательные сети (как было показано ранее, этот подход является новым применительно
к решению поставленной проблемы). Однако же, этот факт не налагает ограничений на используемую топологию сетей, поскольку генеративно-состязательный
подход является концепцией, а не архитектурой, и может быть применен к сетям с любой топологией.

\subsection{Проектирование системы хранения данных}
Система хранения данных --- это комплекс аппаратных и программных средств, который предназначен для хранения и оперативной обработки информации, 
как правило, большого объема \cite{datastorage}. В рамках решения поставленной задачи под системой хранения данных будут подразумеваться преимущественно
программные средства.

Поскольку "<сырая"> (необработанная) информация в рамках поставленной проблемы является неструктурированной (текстовые документы), отдельным
пунктом стоит выделить этап ее структурирования. Под структурированием подразумевается обработка информации таким образом, чтобы необходимые
параметры документов, как самостоятельные, так и по отношению с поисковым запросам, могли быть получены путем запроса к структурированному
источнику (базе) данных.

Выдвинем требования к системе хранения данных:
\begin{enumerate}[1)]
    \item система должна хранить структурированную информацию о документах из поисковой базы, необходимую для работы алгоритмов оценки релевантности;
    \item получение результатов по запросам (например, параметрических характеристик результатов поискового запроса) должно осуществляться за 
          приемлемое время (поскольку обучение нейронных сетей требует многократной генерации поисковых запросов и осуществления фактического поиска
          по ним в базе данных);
    \item заметим, что вопрос занимаемого дискового пространства не столь критичен в связи с тем, что поисковые базы по текстовым документам
          (в частности, наборы данных, применяемые в данном исследовании) занимают сравнительно небольшой объем по сравнению с информацией
          нетекстового характера (изображения, аудиофайлы и тому подобное).
\end{enumerate}

Одним из наиболее универсальных методов структурирования в информационном поиске является составление инвертированного индекса.
Инвертированный индекс (англ. inverted index) --- структура данных, в которой для каждого слова коллекции документов в соответствующем списке 
перечислены все документы в коллекции, в которых оно встретилось. Инвертированный индекс используется для поиска по текстам 
\cite{baezayates99, 10.1145/296854.277632}. 

Инвертированный индекс может быть, в частности, представлен в виде реляционной базы данных, в которую входят следующие таблицы:
\begin{enumerate}[1)]
    \item таблица документов (идентификатор, название, длина, по необходимости - и некоторые другие параметры);
    \item таблица слов (идентификатор, слово, семантические или синтаксические метаданные (например, начальная форма));
    \item собственно, таблица-индекс (идентификатор слова, идентификатор документа, позиция);
    \item вспомогательные таблицы (например, указание встречаемости того или иного слова в каждом документе и тому подобное).
\end{enumerate}

\subsection{Реализация программной системы и ее модулей}
Под реализацией, как уже было описано выше, подразумевается не только ее экспериментальная часть (написание кода программ, реализующих необходимые
алгоритмы, а также их отладка и проверка), но и теоретические методы. К таковым относятся:
\begin{enumerate}[1)]
    \item проектирование архитектуры разрабатываемой программной системы;
    \item анализ существующих библиотек, реализующих основные нейросетевые алгоритмы;
    \item подбор библиотек, наиболее полно и адекватно (применительно к поставленной проблеме) реализующих необходимые подзадачи;
    \item выявление алгоритмов, не покрываемых библиотеками.
\end{enumerate}
Экспериментальная (практическая) часть включает в себя
\begin{enumerate}[1)]
    \item реализацию вспомогательных алгоритмов, не входящих в используемые библиотеки (при необходимости);
    \item собственно реализацию алгоритмов решения поставленной задачи;
    \item отладку написанных программ, проверку корректности реализации алгоритмов.
\end{enumerate}

В качестве основного языка программирования для реализации программной системы был выбран язык Python \cite{CS-R9526, 10.5555/1593511, python}.
Данный выбор обусловлен следующими причинами:
\begin{enumerate}[1)]
    \item это наиболее популярный \cite{ml-popularity-2017} язык для реализации алгоритмов машинного обучения, для которого разработано большое 
          количество библиотек, реализующих нейросетевые алгоритмы;
    \item наличие репозитория библиотек PyPI \cite{pypi}, облегчающего установку и подключение необходимых библиотек, а также позволяющего в большинстве
          случаев избежать проблемы несовместимости версий различных библиотек между собой;
    \item простота синтаксиса языка, позволяющая при разработке алгоритма сосредоточиться на его проектировании (написании необходимых команд);
    \item с другой стороны, язык Python достаточно просто интегрируется с языками низкого уровня, в частности, Си \cite{python-c-api}, что позволяет
          реализовать части программ, критичные к производительности, на подобных языках.
    \item наличие у автора исследования опыта в программировании на данном языке, в том числе с использованием нейросетевых библиотек.
\end{enumerate}

\subsection{Обучение нейронных сетей и получение результатов}
Обучение нейронных сетей --- процесс по определению экспериментальный, однако же, необходимо учитывать следующие факторы:
\begin{enumerate}[1)]
    \item возможность переобучения и недообучения модели \cite{10.1111/j.1751-5823.2011.00149, burnham2003model}. Переобучение (оверфитинг, 
          англ. overfitting) --- явление, когда построенная модель хорошо объясняет примеры из обучающей выборки, но относительно плохо работает 
          на примерах, не участвовавших в обучении (на примерах из тестовой выборки). Это связано с тем, что при построении модели (иными словами --- 
          в процессе ее обучения) в обучающей выборке обнаруживаются некоторые случайные закономерности, которые отсутствуют в генеральной совокупности.
          Поэтому, в связи с вышесказанным, обучающая и тестовая выборки должны иметь примерно одинаковую природу. С другой стороны, недообучение
          (андерфитинг, англ. underfitting) --- явление, при котором модель недостаточно хорошо объясняет особенности генеральной совокупности. Такое
          может наблюдаться в том случае, когда обучающая выборка либо недостаточно велика, либо недостаточно разнообразна по сравнению с генеральной
          совокупностью.
    \item контроль процесса обучения. Это необходимо в связи с тем, что результаты обучения нейронной сети зависят не только от конфигурации 
          нейросетевой модели и входных данных, но также и от самого процесса обучения (разбиение процесса на эпохи, коэффициент скорости обучения
          и тому подобное). Поэтому здесь также необходимо подбирать наиболее адекватную конфигурацию процесса.
\end{enumerate}

\subsection{Сопоставление результатов и выводы}
Сопоставление результатов включает в себя следующие пункты:
\begin{enumerate}[1)]
    \item сравнение результатов, полученных различными моделями (на одних и тех же исходных данных);
    \item сравнение результатов, полученных в данном исследовании, с результатами предшественников (существующими классическими и нейросетевыми алгоритмами).
\end{enumerate}

Сравнение результатов может проводиться как субъективным, так и объективным путем. Субъективное сравнение применительно к поставленной задаче означает
прежде всего "<ручную"> оценку результатов --- а именно, насколько они на самом деле релевантны соответствующим запросам. Однако же, такое сравнение
крайне трудно провести в случае сходных результатов, кроме того, его необходимо проводить беспристрастно, незаинтересованной стороной.

Поэтому куда более важную роль в сопоставлении играют объективные методы сравнения --- то есть, получение количественных характеристик того,
насколько полученные результаты приемлемы, в частности:
\begin{enumerate}[1)]
    \item определение (статистическая оценка) вероятности ошибок первого (когда предположительно нерелевантный запрос попадает в релевантную
          выборку, занимая высокое место при ранжировании) и второго (наоборот, когда релевантный результат признается нерелевантным) рода; 
    \item оценка быстродействия нейросетевых алгоритмов по сравнению с классическими (насколько приемлема их скорость работы);
    \item проверка и сопоставление порядка ранжирования результатов в поисковой выдаче.
\end{enumerate}

На основании проведенного сопоставления делаются следующие выводы:
\begin{enumerate}[1)]
    \item о целесообразности применения подхода с использованием генеративно-состязательных сетей в принципе, на концептуальном уровне;
    \item о качестве конкретных разработанных моделей и возможности их практического применения;
    \item об ограничениях предложенного метода;
    \item о вопросах, требующих дальнейшего исследования.
\end{enumerate}