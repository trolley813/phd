Проблемы информационного поиска в настоящее время представляют значительный интерес, как коммерческий, так и академический. Объемы обрабатываемой
и хранимой информации постоянно растут, в связи с чем остро стоят проблемы ее структурирования и каталогизации. В сети Интернет, без которой сейчас
уже невозможно представить существование современного мира, добыча и отбор нужной информации происходит практически исключительно средствами 
информационного поиска. 

В связи со всем вышесказанным, существенную важность приобретает релевантность результатов поисковой выдачи, поскольку от нее напрямую зависит
качество применяемой поисковой системы, а также ее конкурентоспособность. Будучи мерой важности результатов выдачи поисковой системы для 
ее пользователя --- человека, релевантность так или иначе является субъективной характеристикой. Однако же, очевидно, что она подлежит оценке
автоматизированными методами и алгоритмами.

Следовательно, мы можем говорить об актуальности проблемы оценки релевантности.

Сравнительно недавно (середина 2010-х годов) в теории нейронных сетей появилась концепция генеративно-состязательных сетей, которые позволяют 
создавать объекты, в той или иной мере "<похожие"> на заданные. Данная концепция нашла свое применение для разнообразного спектра задач --- от 
генерации изображений и фотографий до применения в теории игр. Однако, в сфере информационного поиска описанный выше подход пока не получил
широкого распространения, хотя он и упоминается в некоторых научных работах. Таким образом, следует считать необходимым проведение
дальнейшего научного поиска в направлении нейросетевого подхода с использованием генеративно-состязательных сетей.


В связи с вышеизложенным, сформулируем цель исследования --- разработать методы оценки релевантности результатов поиска при 
условии использования нейросетевого подхода, проанализировав при этом существующее положение данной проблемы и методы ее решения.

В данной работе была показана принципиальная осуществимость отбора релевантных результатов поиска нейросетями при использовании
корпуса текстов общего назначения и запросов, генерируемых другой нейросетью. Вышеизложенное составляет научную новизну исследования.

Один из методов решения проблемы был представлен автором на конференции East \& West Design \& Test 2020 в сентябре 2020 года. 
Научная статья была опубликована в материалах конференции \cite{9224840}. % TODO: Место для ссылки на статью

Структурно диссертация состоит из трех глав. В первой главе рассматривается современное состояние проблемы, анализируются
научные работы современных исследователей по теме проблемы и смежным с ней темам, после чего делается вывод о необходимости
проведения собственных исследований по теме. 

Во второй главе проблема рассматривается с теоретической точки зрения --- вводятся необходимые понятия и определения, после чего 
проблема разбивается на подзадачи. Далее следует анализ каждой из подзадач в плане методов и алгоритмов их решения.

Наконец, третья глава содержит описание результатов исследования. В ней рассмотрены подробности и детали практической реализации
моделей решения проблемы, полученные при этом результаты, а также из анализ и обсуждение. Здесь также рассматриваются вопросы,
требующие дальнейшего исследования.
