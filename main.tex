\documentclass{memoir}
\usepackage{tempora}
\usepackage[russian]{babel}
\usepackage[utf8]{inputenc}
\usepackage{csquotes}
\usepackage{indentfirst}
\usepackage[bibstyle=gost-numeric,citestyle=gost-numeric]{biblatex}
\usepackage[left=25mm, top=20mm, right=10mm, bottom=20mm]{geometry}
\usepackage[14pt]{extsizes}
\usepackage[bigdelims,vvarbb]{newtxmath}
\usepackage{graphicx}
\DeclareBibliographyCategory{cited}
\AtEveryCitekey{\addtocategory{cited}{\thefield{entrykey}}}
\newtheorem{defn}{Определение}
\renewcommand{\baselinestretch}{1.5}
\bibliography{biblio}
\nocite{*}
%\DeclareUnicodeCharacter{0301}{*************************************}
\begin{document}
\tableofcontents
\chapter*{Введение}
\input{000_intro.tex}
\chapter{Современное состояние проблемы оценки релевантности}
\section{Обзор существующих публикаций}
%!TEX encoding = UTF-8

Ниже приводится обзор статей на тему использования методов машинного обучения в системах информационного поиска и, в частности, 
их применения к оценке релевантности, опубликованных за последнее время (2016--2020 годы).

В статье \cite{DBLP:journals/corr/MitraC17} описываются различные архитектуры сетей глубинного обучения применительно к задачам
информационного поиска. Авторы статьи затрагивают в исследовании задачи как распознавания поискового запроса, так и оценки 
релевантности результатов. Однако, для последних описывается преимущественно подход, основанный на обучении "<с учителем">, когда
метки релевантности/нерелевантности запросов заданы заранее (ground truth). Генеративно-состя\-зательные сети в данной работе
не рассматриваются.

Статья \cite{DBLP:journals/corr/abs-1802-10078} описывает нейросетевую архитектуру поисковой системы, предназначенной для
текстов узкой направленности (научных статей в области биологии и медицины). В работе описана так называемая дельта-модель, 
состоящая из сверточной подсети, за которой следует подсеть с прямой связью (feed-forward). Сверточная модель использует метрики
схожести (similarity), основанные на учете $n$-грамм.

В качестве обучающей выборки в \cite{DBLP:journals/corr/abs-1802-10078} используются журналы кликов (click logs) поисковых систем, 
которые в той или иной мере позволяют судить о релевантности полученных пользователями результатов, то есть, являются своего рода
метками релевантности. Таким образом, можно говорить о схожести методов, предлагаемых в \cite{DBLP:journals/corr/MitraC17} и 
\cite{DBLP:journals/corr/abs-1802-10078}.

Статья \cite{DBLP:journals/corr/abs-2001-09896} посвящена семантической оценке важности слов в контексте документа (вариации 
TF-IDF). На данный момент (28.02.2020) статья не завершена, однако в ней присутствуют необходимые результаты и их оценка, которая
заключается в сравнении семантической вариации с "<классической">. Несмотря на то, что авторы применяют полностью алгоритмические
методы, без использования нейросетей, статья представляет интерес для темы работы.

Публикация \cite{DBLP:journals/corr/abs-2001-07075} рассматривает проблему оценки релевантности в информационном поиске несколько
с другой стороны --- использования квантовоподобной (quantum-like) структуры для суждений о релевантности. Авторы сравнивают
предложенную ими квантово-вероятностную модель с байесовской. Опять же, нейросетевые алгоритмы авторами не рассматриваются,
но, тем не менее, данный метод также представляет существенный интерес.
 
В статье \cite{DBLP:journals/corr/abs-1910-00314} рассматриваются задачи оценки релевантности запросов применительно к узкой
предметной области --- биологии, медицины и здравоохранения. В качестве документов авторы рассматривают аннотации к публикациям,
размещенным в системе PubMed. В публикации описываются различные способы представления запросов, документов и предложений, включая
и TF-IDF (аналогично \cite{DBLP:journals/corr/abs-2001-09896}). Авторы выделяют 2 основные задачи --- нейросетевой модели для
ранжирования (с использованием SVM для первичного ранжирования и "<классических"> функций по типу BM25 для повторного), а также
построения моделей, использующих множество представлений (multi-view), таких, как TF-IDF и модель "<мешка слов"> (bag-of-words).
GAN-сети в данной работе не рассматриваются, однако, во второй поставленной авторами задаче используются в том числе и методы
обучения без учителя.

Имеет смысл упомянуть и работу \cite{DBLP:journals/corr/abs-1909-06859}, в которой предлагается модель MarlRank многоагентного
обучения ранжированию с подкреплением (конкретные архитектуры нейросетей авторами не описываются). В данной работе авторы 
рассматривают каждый документ как агент в марковском процессе принятия решений. Предсказание релевантности документом 
осуществляется на основе как его собственных характеристик, так и характеристик схожих документов, в связи с чем (оценка 
релевантности на основе конечного набора характеристик) публикация также включена в данный обзор как представляющая интерес. 

Статья \cite{DBLP:journals/corr/abs-1907-08657} поднимает проблему ограниченной доступности меток релевантности запросов,
оцениваемых непосредственно пользователями-экспертами (а эта проблема является довольно актуальной для поставленных задач).
Авторами предлагаются методы повышения такой доступности путем генерации подобных меток и их отбора с использованием в 
том числе нейросетевых алгоритмов, в связи с чем данная работа также довольно близка поставленным задачам. Опять-таки, 
генеративно-состя\-зательные сети в публикации не упомянуты.

Работа \cite{DBLP:journals/corr/abs-1908-06132} представляет собой кандидатскую (PhD) диссертацию соискателя, представляющего
один из университетов штата Нью-Йорк (США). Автор подробно рассматривает различные нейросетевые модели и их применение в информационном
поиске. Тема генеративно-состязательных сетей, равно как и тема нейросетевой оценки релевантности, не затрагивается.

Работа \cite{DBLP:journals/corr/abs-1906-09404} предлагает нейросетевую архитектуру RLTM(Reinforced Long-Text Matching), предназначенную для эффективного 
ранжирования запросов к «длинным» документам. Хотя упоминаемые в работе архитектуры сетей не являются генеративно-состязательными
по своему принципу, в них также применяется принцип различения (дискриминации) между положительными и отрицательными результатами
(основанный на ненейросетевом подходе). Авторы также предлагают подход по отбору из текстов документов наиболее значимых
предложений, указывая критерии такой значимости. 

Предметная область в публикации не указана, в качестве используемых наборов данных используются результаты поисковых запросов к
системам общего назначения (на китайском языке).

В публикации \cite{DBLP:journals/corr/abs-1904-06808} предлагается аксиоматический подход к регуляризации нейросетевых моделей 
ранжирования. Хотя данная работа не относится напрямую к теме исследования, тем не менее, описанные в ней принципы и подходы 
могут оказаться полезными при разработке нейросетевых архитектур применительно к поставленным задачам.

Статья \cite{DBLP:journals/corr/abs-1903-06902} представляет собой достаточно объемный обзор нейросетевых моделей оценки 
релевантности и ранжирования результатов. Авторами рассматриваются как архитектуры моделей ранжирования, так и эмпирические
их оценки. Тем не менее, GAN-сети в работе не упоминаются, что может свидетельствовать о существенной научной новизне
исследования этой области.

Публикация \cite{DBLP:journals/corr/abs-1812-00073} посвящена библиотеке TF-Ranking, представляющей собой дополнение для 
популярного набора TensorFlow, которое предназначено для обучения ранжированию. Несмотря на то, что статья является 
по содержанию преимущественно технической, она представляет интерес в плане реализации подобных нейросетевых архитектур 
в качестве дополнений к существующим библиотекам машинного обучения.

Также интерес представляет статья \cite{DBLP:journals/corr/abs-1810-12936}, которая предлагает нейросетевые модели 
для подхода псевдообратной связи по релевантности (pseudo relevance feedback). Авторы отмечают сложности, возникающие при
комбинировании PRF с нейросетевыми моделями и в связи с этим предлагают единый каркас, объединяющий данные подходы.

В работе \cite{DBLP:journals/corr/abs-1809-01682} описывается глубинное ранжирование по релевантности (deep relevance ranking).
Авторы подробно рассматривают ряд архитектур на основе модели DRMM, их применимость к задачам ранжирования, а также проводят
тестирование созданных моделей на наборе данных BIOASQ для автоматизированных ответов на вопросы (question answering).
Данная тема является смежной к теме оценки релевантности поисковых запросов (если рассматривать заданный пользователем
вопрос в качестве поискового запроса, а ответ в качестве результата).

Публикация \cite{DBLP:journals/corr/abs-1807-05355} напрямую не относится к теме исследования, но, тем не менее, поднимает
любопытный вопрос — влияние психологического эффекта порядка (в котором пользователю представлены результаты поиска)
на многомерные системы суждения о релевантности на основе журналов запросов (query logs).

Значительный интерес для поставленной темы представляет статья \cite{DBLP:journals/corr/abs-1806-03577}. Данная публикация
является единственной, где поднимается тема генеративно-состязательных сетей применительно к задачам информационного поиска.
Основной рассматриваемой задачей является генерация поисковых запросов, методы же оценки релевантности с помощью GAN-сетей 
в работе не рассмотрены, в качестве модели автор предлагает вышеописанную PRF (как вариант, основанную на нейросетевой 
архитектуре).

Несмотря на то, что публикация \cite{DBLP:journals/corr/abs-1806-03577} носит преимущественно обзорный характер, не имея
практически никаких технических подробностей (однако, в ней приведены ссылки на используемое и упоминаемое в тексте 
программное обеспечение), ее следует считать одним из основных источников при работе над темой исследования.

Также к данной теме относится работа \cite{DBLP:journals/corr/abs-1805-02184}, в которой рассматривается моделирование 
многомерной релевантности с использованием векторных пространств. В контексте поставленных задач, статью можно отнести
к смежным, поскольку в ней рассматриваются исключительно алгоритмические (ненейросетевые) методы решения, и основная тема
статьи почти не связана с оценкой самой релевантности, однако данный материал следует считать ценным для исследования.

Публикация \cite{DBLP:journals/corr/abs-1711-08611} рассматривает DRMM для поиска по произвольному запросу (ad-hoc retrieval)

В статье \cite{DBLP:journals/corr/abs-1710-05649} предлагается модель DeepRank для глубинного обучения ранжированию, 
симулирующая процесс оценки релевантности человеком. Архитектурно, она состоит из модуля, извлекающего контексты для оценки
релевантности, нейросети (сверточной, либо рекуррентной) для определения «локальных» релевантностей, а также агрегирующей
сети для вычисления «глобальной» оценки (по всему документу в целом). Согласно исследованию авторов, метод показывает высокие
результаты, сравнимые с текущими моделями глубинного обучения и обучения ранжированию, а иногда и превосходящие их.
Генеративно-состязательные сети в работе непосредственно не упоминаются, однако, для темы исследования данная публикация 
также представляет интерес.

В работе \cite{DBLP:journals/corr/abs-1709-01709} затрагивается тема отбора результатов (семплинга, англ. sampling)
для широкомасштабной оценки поисковых результатов. Авторы предлагают ряд различных методов для создания коллекций запросов
(такие коллекции в большей степени требуют вмешательства человека, так как они используются для формирования меток
ground truth). Данную публикацию также стоит считать ценным материалом для темы исследования, в связи с тем, 
что определенная часть оценки результатов будет проводиться субъективно.
\chapter{Методы исследования}
Введем несколько определений, используемых в дальнейшем.
\begin{defn}
    Генеративно-состязательная сеть (GAN) - это нейронная сеть, основной целью которой является генерация объектов,
    сходных с таковыми из заданного набора (по некоторой метрике) \cite{10.5555/2969033.2969125}.
\end{defn}
Это поведение реализуется с помощью следующей архитектуры:
\begin{itemize}
    \item генерирующая сеть $G$ (генератор) создает (генерирует) объекты заданной структуры.
    \item различающая сеть $D$ (дискриминатор) сопоставляет сгенерированные объекты с набором эталонных (ground-truth) значений,
          делая выводы об их сходстве. Сеть G обучается на основе обратной связи, полученной от сети D (с использованием обычных
          методов обратного распространения ошибок).
\end{itemize}
Генеративно-состязательные сети относятся к обучению без учителя. Выставление меток в обучающем наборе не требуется,
единственной требуемой частью является набор эталонных значений.

Следует обратить внимание, что GAN --- это в первую очередь концепция, нежели архитектура, а это означает, что
генеративно-состязательный подход может быть использован с любой сетевой архитектурой (например, многослойными
персептронами~\cite{Rosenblatt1958ThePA}, LSTM-сетями~\cite{10.1162/neco.1997.9.8.1735} и т. д.). В данной работе
концепция GAN используется с многослойными сетями с прямой связью.
Типы сетей, представленные здесь, --- это полностью связанная сеть типа персептрона и сеть, имеющая несколько слоев для отсева
(drop-out).

Введем также несколько понятий из теории информационного поиска, в первую очередь понятия частоты слова и обратной частоты документа.

\begin{defn}
    Метрика $TF$ (term frequency --- частота слова) --- отношение числа вхождений некоторого слова к общему числу слов документа.
\end{defn}
Таким образом, $TF$ служит как бы оценкой важности того или иного слова в пределах конкретного документа.
В качестве формул для вычисления $TF$ чаще всего используются \cite{manning_raghavan_schuetze_2008}
непосредственное количество вхождений слова
\begin{equation}
    \label{eq:raw-tf}
    \text{TF}(w, d) = N_w(d)
\end{equation}
и нормированное количество
\begin{equation}
    \label{eq:norm-tf}
    \text{TF}(w, d) = \frac{N_w(d)}{N(d)},
\end{equation}
где $\text{TF}(w, d)$ --- метрика $TF$ применительно к слову $w$ в документе $d$, $N_w(d)$ --- количество вхождений слова $w$
в документ $d$, а $N(d)$ --- общее количество слов в документе $d$ (иными словами, длина документа $d$, выраженная в словах).

Метрика $IDF$ была впервые введена Карен Спарк Джонс в \cite{jones2004statistical}. Она определяется таким образом:
\begin{defn}
    Метрика IDF (inverse document frequency --- обратная частота документа) --- инверсия частоты, с которой некоторое слово
    встречается в документах коллекции.
\end{defn}
Таким образом, $IDF$ показывает, насколько то или иное слово распространено в документах коллекции. Чем более широко употребительным
является слово, тем меньше его $IDF$. Чаще всего данная метрика используется в связке с $TF$ в качестве весовой.

В большинстве случаев IDF определяется по формуле
\begin{equation}
    \label{eq:idf}
    \text{IDF}(q_i) = \log \frac{N - n(q_i) + 0.5}{n(q_i) + 0.5},
\end{equation}
где $N$ --- общее количество документов в коллекции, а $n(q_i)$ --- количество документов, содержащих слово $q_i$
(данная формула используется в алгоритме BM25 \cite{Amati2009}). Однако, определение \eqref{eq:idf} обладает существенным недостатком:
наиболее употребительные слова, а именно, встречающиеся более чем в половине документов из всей коллекции, будут обладать
отрицательными $IDF$. Существует несколько подходов для устранения этого недостатка. Наиболее простым является введение в формулу
слагаемого сдвига:
\begin{equation}
    \label{eq:shifted-idf}
    \text{IDF}(q_i) = \log \frac{N - n(q_i) + 0.5}{n(q_i) + 0.5} - \log\frac{0.5}{N + 0.5},
\end{equation}
тогда IDF слова, содержащегося во всех без исключения документах коллекции, будет равна нулю, а в противном случае (если есть
хотя бы один документ, не содержащий данного слова) она будет строго положительной.

Перейдем к определениям, касающимся семантического анализа (в первую очередь, из дистрибутивной семантики).
\begin{defn}
    Коэффициент семантической близости, реже семантического подобия (semantic similarity) --- метрика (скаляр), определенная
    на множестве документов, определяющая <<расстояние>> между словами на основании сходства значений этих слов.
\end{defn}

Семантическое подобие может определяться как на основании фактических данных (например, тезауруса --- словаря синонимов),
так и с помощью средств дистрибутивной семантики --- области лингвистики, которая занимается вычислением степени
семантической близости между лингвистическими единицами на основании их распределения (дистрибуции) в больших массивах
лингвистических данных (текстовых корпусах). В настоящей работе рассматриваются оба определения.

\begin{defn}
    Векторное представление слов --- общее название для различных подходов к моделированию языка и обучению представлений
    в обработке естественного языка, направленных на сопоставление словам (и, возможно, фразам) из некоторого словаря
    векторов из $\mathbb{R}^{n}$ для $n$, значительно меньшего количества слов в словаре.
\end{defn}
\section{Разбиение проблемы на задачи}
Проблема разработки методов оценки релевантности с использованием нейросетевого подхода может быть
разбита на подзадачи следующим образом:
\begin{enumerate}[1)]
    \item Создание базы поисковых запросов. Данная база является необходимой как для формирования эталонного
    набора релевантных (псевдорелевантных) запросов, требуемого для обучения нейронных сетей, так и для последующего
    тестирования и оценки качества получаемой модели.
    \item Анализ конфигураций нейронных сетей для модели оценки релевантности. В зависимости от выбранной модели,
    ее параметризации и т. д. могут потребоваться различные конфигурации и топологии нейросетей. 
    \item 
\end{enumerate}
\chapter{Результаты исследования}
\section{Описание выполненных экспериментов}
В данной работе рассматриваются следующие модели для построения нейросетей, оценивающих релевантность:
\begin{enumerate}[1)]
    \item Модель, основанная на грамматике текста. Данная модель является одной из наиболее простых и может
    рассматриваться как первый шаг на пути к семантике. Структурной единицей текста при использовании такой модели
    является слово (с отбрасыванием всевозможных грамматических форм, флексий и~т.~п.)
\end{enumerate}
\subsection{Модель с учетом грамматики}
\subsubsection{Основная архитектура программной системы}
В данной работе предлагается следующая архитектура системы:
\begin{itemize}
\item модуль генерации поисковых запросов, предназначенный для обучения оценивающей части. Данный модуль предлагается
реализовать на нейросетевой основе, используя GAN-подобную архитектуру либо GPT-2;
\item поисковый модуль.
\item модуль формализации результатов, сопоставляющий каждой паре "<запрос-документ"> $\left\langle q, d \right\rangle$ набор числовых параметров
$\{p_i\}$, представляющих собой характеристики результата поискового запроса. Предлагается организовать вычисление данных характеристик на основе
классических алгоритмов без использования нейронных сетей;
\item модуль оценки релевантности, принимающий на вход набор параметров $\{p_i\}$ и выдающий результат оценки. Для данной части, представляющей 
наибольший научный и практический интерес, также предлагается использовать генеративно-состязательную нейросетевую архитектуру.
\end{itemize}
Опишем подробно каждый модуль в отдельности.

Модуль генерации поисковых запросов. Для обучения модуля оценки релевантности требуется генерировать большой объем данных (числовых векторов $\{p_i\}$),
однако, использовать случайные числа для этой цели нецелесообразно в связи с тем, что вероятностное распределение таковых может оказаться существенно
отличным от получаемого при реальных поисковых запросах. По причине того, что задача исследования распределения вероятностей требует существенного анализа,
а ручная генерация поисковых запросов в достаточных для решения основной задачи количествах технически неосуществима, более простым представляется путь 
генерации самих запросов с использованием нейронных сетей.

Данную подзадачу также возможно решать с помощью GAN-сетей, однако более перспективным представляется путь использования сети GPT-2, созданной в OpenAI.
Данная архитектура сети проектировалась авторами специально для решения задачи генерации текстов, которыми в задаче данной работы являются поисковые запросы.

Поисковый модуль. Данная часть программной системы осуществляет поиск по заданному запросу в подготовленной базе данных. В самом простом случае она
представляет собой инвертированный индекс, однако, возможно подключение, например, семантических средств (антологий, тезаурусов и т. п.) Модуль работает с
использованием классических алгоритмов, однако возможно подключение и нейронных сетей для оптимизации результатов поиска (путем хранения истории запросов
определять потенциально наиболее релевантные результаты, анализируя схожесть запросов).

Модуль формализации результатов. Служит для вычисления характеристик поискового результата (пары "<запрос-документ">), необходимых для определения его 
релевантности нейронной сетью. К числу данных характеристик, в частности, могут относиться такие параметры, как встречаемость в документе слов из запроса
либо их сочетаний, с грамматической точки зрения --- учет различных словоформ, с семантической --- учет синонимов слова в контексте запроса и т.п.
Данный модуль также реализован с использованием только классических алгоритмов, без подключения нейронных сетей.

Модуль оценки релевантности. Данная часть программной системы, как говорилось выше, представляет собой наибольший интерес, как академический, так
и практический. Здесь она реализована с помощью генеративно-состязательной сети, в которой:
\begin{itemize}
\item подсеть $G$ генерирует векторы параметров из созданных поисковых запросов на основании расчета их модулем формализации. Обучение подсети осуществляется
с помощью обратной связи от подсети $D$;
\item в свою очередь, подсеть $D$ отбирает релевантные результаты путем сравнения с базой данных заведомо релевантных запросов. В качестве источника таковых,
помимо ручного отбора результатов, представляет интерес смешанный подход, в котором определенная доля эталонов генерируется с помощью классических функций
оценки релевантности (BM25 и т.п.). Данный метод позволяет увеличить используемую выборку без повышения трудозатрат на ручную обработку, однако вопрос
качества получаемых результатов требует дальнейшего исследования.
\end{itemize}
После обучения модель пригодна к использованию для произвольных запросов, что позволяет субъективно оценить получаемые результаты.

\subsubsection{Источники текстовых документов}
В качестве корпуса текстов в реализации был использован набор данных "<Гутенберг"> (Gutenberg Dataset) ~\cite{lahiri:2014:SRW}. 
Он представляет собой собрание из 3036 книг, написанных (на английском языке) 142 авторами, преимущественно 19 века. 
Набор данных состоит из простых текстовых файлов, очищенных от любых метаданных, лицензионной информации и заметок переписчиков, 
насколько это возможно. Данные условие облегчают анализ текстов, так как позволяют избавиться от стадий предварительной обработки
и тем самым упростить задачу обучения нейросетей на текстах.

\subsubsection{База данных}
База данных была реализована с использованием СУБД ClickHouse, разработанного отечественной компанией "<Яндекс">
\cite{Clickhouse2020}. Данная СУБД обеспечивает быстрые запросы выборки (SELECT) по таблицам с большим (величиной порядка $10^9$)
количеством строк, что очень важно для решения поставленной задачи. С другой стороны, ClickHouse не подходит ни для запросов
обновления/удаления (UPDATE и DELETE), ни для "<тяжелых"> объединяющих запросов (JOIN). Однако данный факт 
не вносит большого вклада в реализацию, поскольку база слабо реляционна по своей природе. Таким образом, использование 
ClickHouse в качестве хранилища является разумным выбором.

Тексты были предварительно проанализированы с помощью лексического анализатора Penn Treebank~\cite{10.3115/1075812.1075835}
для разбора слов. Впоследствии слова были обработаны стеммером Портера~\cite{Porter1980AnAF}, чтобы восстановить основы слов, 
объединив в одной записи всевозможные грамматические формы одного слова.
Каждой основе слова, встречающемуся в текстовом корпусе, был присвоен числовой идентификатор, а идентификаторы 
с соответствующими основами хранились в SQL-таблице.

Инвертированный индекс в базе данных хранится в следующей форме (DDL-запрос):

\begin{verbatim}
    CREATE TABLE default.inv_index
    (
        `word_id` Int32,
        `document_id` Int32,
        `start_pos` Int32,
        `end_pos` Int32
    )
    ENGINE = MergeTree()
    PARTITION BY document_id
    ORDER BY document_id
    SETTINGS index_granularity = 8192
\end{verbatim}

Таким образом, таблица секционируется (разбивается на разделы-"<партиции">) по идентификатору документа, чтобы разрешить
более быстрые запросы \texttt{SELECT}, связанные с одним документом (что часто происходит при выполнении поиска).

Инвертированные индексы для биграмм и триграмм используют в основном ту же структуру, за исключением нескольких столбцов
\texttt{word\_id}.

Еще одна особенность ClickHouse, которая широко используется здесь, --- это возможность создавать материализованные представления.
Это в основном представления SQL с кэшированным результатом, хранящимся на диске. Они особенно полезны, например, при запросе
количества слов в документе (чтение кэшированных данных дает значительное, иногда на порядок, ускорение).

\subsubsection{Генерация поисковых запросов}
Генератор запросов основан на реализации сети GPT-2 с помощью Python-пакета \texttt{textgenrnn}~\cite{TextgenRNN2020}. Данная 
реализация написана с использованием фреймворка Keras~\cite{chollet2015keras}. На более низком уровне используется библиотека
TensorFlow~\cite{tensorflow2015-whitepaper}, в которой для ускорения обучения сети путем расчетов на графическом процессоре
применяется cuDNN~\cite{DBLP:journals/corr/ChetlurWVCTCS14} --- библиотека, разработанная компанией nVIDIA специально для 
обучения нейронных сетей с помощью CUDA~\cite{cuda}. Топология сети в пакете \texttt{textgenrnn} представлена на рис.~\ref{fig:default}

\begin{figure}
    \centerline{\includegraphics[scale=0.5]{default_model.png}}
    \caption{Структура модели нейронной сети в \texttt{textgenrnn}}\label{fig:default}
\end{figure}

Чтобы сгенерированные запросы были потенциально релевантны пространству поиска, сеть GPT-2 должна быть обучена на одном и том же
наборе текстов. Чтобы избежать более длительного времени обучения, но вместе с тем сохранить качество и отношение к предметной области,
для обучения рекуррентной сети была выбрана доля набора данных в размере 5\% (всего 152 документа и около 13 миллионов слов).

Кроме того, чтобы результирующие запросы не были слишком универсальными (например, содержащими только "<стоп-слова">, такие как
артикли, предлоги и другие часто используемые слова), на генератор накладываются некоторые ограничения. А именно, запрос считается
"<хорошим"> тогда и только тогда, когда он удовлетворяет одному из следующих условий:

\begin{enumerate}[1)]
    \item как минимум 1 слово встречается не более чем в 25\% от всех документов из коллекции;
    \item как минимум \(\frac{1}{3}\) слов (с округлением до ближайшего целого числа) встречается
    не более чем в 40\% от всех документов из коллекции.
\end{enumerate}

В общей сложности, в качестве эталонных было выбрано 242 запроса (на английском языке).

Примеры сгенерированных запросов:
\begin{itemize}
    \item \textit{returned earl   we had }
    \item \textit{trouble to move in oz mode which}
    \item \textit{and i don really want you to take}
    \item \textit{literature seems to be the loveliest ends }
    \item \textit{your distinction  it is i cannot as}
\end{itemize}

Как видим (по крайней мере, с точки зрения неносителя языка), запросы вполне сходны по структуре с "<настоящими"> поисковыми
запросами к системам общего назначения, таким, как Google.

Далее, в качестве эталона псевдорелевантности для обучения нейронной сети были отобраны по 10 верхних результатов по каждому из
запросов. Результаты были отранжированы согласно модифициорванной формуле BM25:
\begin{equation}
    \label{eq:wbm25}
    \begin{aligned}
    \text{score}(D,Q) = w_1 \sum_{i=1}^{n} \text{IDF}(q_i) \times \\ \times \frac{f(q_i, D) \cdot (k_1 + 1)}{f(q_i, D) + k_1 \cdot \left(1 - b + b \cdot \frac{|D|}{\tilde{L}}\right)} \\
    + w_2  \sum_{i=1}^{n-1} \text{IDF}(q_i q_{i+1}) \times \\ \times \frac{f(q_i q_{i+1}, D) \cdot (k_1 + 1)}{f(q_i q_{i+1}, D) + k_1 \cdot \left(1 - b + b \cdot \frac{|D| - 1}{\tilde{L} - 1}\right)} \\
    + w_3  \sum_{i=1}^{n-2} \text{IDF}(q_i q_{i+1}q_{i+2}) \times \\ \times \frac{f(q_i q_{i+1}q_{i+2}, D) \cdot (k_1 + 1)}{f(q_i q_{i+1}q_{i+2}, D) + k_1 \cdot \left(1 - b + b \cdot \frac{|D| - 2}{\tilde{L} - 2}\right)}
    \end{aligned}
\end{equation}
где \(w_1\), \(w_2\), \(w_3\) --- весовые коэффициенты отдельных слов, биграмм и триграмм соответственно (в реализации 
использовались значения \(w_1=1\), \(w_2=10\) and \(w_3=100\)), \(f(q_1, D)\), \(f(q_1q_2, D)\) и \(f(q_1q_2q_3, D)\) --- 
частоты терминов: отдельного слова \(q_1\), биграммы \(q_1q_2\) и триграммы \(q_1q_2q_3\) соответственно (аналогично и для IDF),
 \(|D|\) --- длина документа \(D\), а \(\tilde{L}\) --- средняя длина документа в коллекции (слагаемые \(-1\) and \(-2\) 
для биграмм и триграмм соответственно были введены в связи с тем, что документ с \(N\) словами, очевидно, содержит 
\(N-1\) биграмму и \(N-2\) триграммы, поэтому "<длина"> документа окажется на один и два меньше соответственно).

Таким образом, эталонный набор данных включал в себя 2158 результатов
(это число меньше, чем $10\times242=2420$, поскольку некоторые запросы дали менее 10 результатов в целом).

\subsubsection{Генеративно-состязательная сеть}
GAN-сеть состояла из следующих подсетей:
\begin{itemize}
    \item подсеть $G$, использующая случайные 100-мерные векторы в качестве входных данных 
    и 32-мерные векторы результатов (метрик) запроса в качестве выходных данных;
    \item подсеть $D$, которая принимает 32-мерные векторы (как порожденные подсетью $G$, так и эталонные) и выводит скаляр, 
    который может быть интерпретирован как "<вероятность"> того, что запрос является псевдорелевантным.
\end{itemize}

В сети использовались следующие функции активации:
\begin{itemize}
    \item гиперболический тангенс
    \begin{equation}
        f(x) = \tanh x;
    \end{equation}
    \item сигмоида
    \begin{equation}
        f(x) = \frac{1}{1+e^{-x}};
    \end{equation}
    \item выпрямитель с "<протечкой"> ~\cite{DBLP:journals/corr/XuWCL15} (leaky rectified linear unit)
    \begin{equation}
        f(x) = \begin{cases}
            x, & x \geqslant 0, \\
            \alpha x, & x < 0.
        \end{cases}
    \end{equation}
\end{itemize}

Внутри подсети $G$ используется последовательное размещение слоев, как показано ниже:
\begin{itemize}
    \item входной слой с размером 100 (для случайного
    входа);
    \item слой с 256 нейронами, с использованием выпрямителя с протечкой при $\alpha = 0.2$;
    \item слой с 512 нейронами, с использованием выпрямителя с протечкой при $\alpha = 0.2$;
    \item слой с 1024 нейронами, с использованием выпрямителя с протечкой при $\alpha = 0.2$;
    \item выходной слой с размером 32, использующий гиперболический тангенс в качестве функции активации.
\end{itemize}
Внутри подсети $D$ также используется последовательное размещение слоев в следующем порядке:
\begin{itemize}
    \item входной слой с размером 32 (для подачи векторов параметров, генерируемых подсетью G);
    \item слой с 1024 нейронами, использующий ReLU с $\alpha = 0.2$;
    \item слой отсева с коэффициентом 0,3;
    \item слой с 512 нейронами, использующий ReLU с $\alpha = 0.2$;
    \item слой отсева с коэффициентом 0,3;
    \item слой с 256 нейронами, использующий ReLU с $\alpha = 0.2$;
    \item слой отсева с коэффициентом 0,3;
    \item выходной слой с размерностью 1 (скалярное значение, обозначающее оценку релевантности).
\end{itemize}

Для проверки сгенерированной модели был сгенерирован дополнительный набор из 51 "<хорошего"> запроса. Затем 
было использовано классическое ранжирование BM25 по формуле \eqref{eq:wbm25}, после чего из результатов поиска 
по каждому запросу были выделены верхние 10 (псевдорелевантные) и нижние 10 (псевдонерелевантные).
На следующем этапе результаты подавались через обученную модель. Некоторые результаты приведены в таблице \ref{tab1}
(средние значения $\mu$ и стандартные отклонения $\sigma$ оценок GAN-сетью для верхней и нижней групп 10), а также
на графике (рис. \ref{fig:gram-scores}), где $S_{\mathrm{ps-rel}}$ и $S_{\mathrm{ps-irr}}$ --- оценки для
псведорелевантных и псевдонерелевантных запросов соответственно.

\begin{figure}
    \centerline{\includegraphics[scale=0.8]{311_scores.eps}}
    \caption{Оценки релевантности для модели}\label{fig:gram-scores}
\end{figure}

\begin{table}[tbp]
    \caption{Результаты поисковых запросов}
    \begin{center}
    \begin{tabular}{ccc}
    \toprule
    \textbf{Запрос}&\multicolumn{2}{c}{\textbf{Оценка релевантности}} \\
    & \textbf{\textit{Top 10}}& \textbf{\textit{Bottom 10}} \\
    \midrule
    that there were no reason on lewis& \(\mu=0.8024\) & \(\mu=0.6926\) \\
    & \(\sigma=0.1050\) & \(\sigma=0.0474\) \\
    \midrule
    i proposed marriage his visit& \(\mu=0.5127\) & \(\mu=0.1933\) \\
    & \(\sigma=0.2786\) & \(\sigma=0.0694\) \\
    \midrule
    fanny  said she& \(\mu=0.7270\) & \(\mu=0.2061\) \\
    & \(\sigma=0.2269\) & \(\sigma=0.0086\) \\
    \midrule
    published it  i think we found& \(\mu=0.2096\) & \(\mu=0.0822\) \\
    & \(\sigma=0.2077\) & \(\sigma=0.0226\) \\
    \midrule
    i shall you heart now & \(\mu=0.4713\) & \(\mu=0.1577\) \\
    & \(\sigma=0.3046\) & \(\sigma=0.0095\) \\
    \midrule
    right i had down her & \(\mu=0.4439\) & \(\mu=0.1843\) \\
    & \(\sigma=0.1391\) & \(\sigma=0.0985\) \\
    \midrule
    nice declaration  she got you  i & \(\mu=0.7382\) & \(\mu=0.3731\) \\
    & \(\sigma=0.2437\) & \(\sigma=0.0203\) \\
    \midrule
    i never moved nearer than it is & \(\mu=0.7801\) & \(\mu=0.2660\) \\
    & \(\sigma=0.1304\) & \(\sigma=0.2241\) \\
    \bottomrule
    \end{tabular}\label{tab1}
    \end{center}
\end{table}
\subsection{Модель с использованием семантической сети}

WordNet \cite{Miller95wordnet:a} --- лексическая база данных английского языка, разработанная в Принстонском университете.
Она представляет собой электронный словарь-тезаурус и набор семантических сетей для английского языка.

Словарь, представленный в WordNet, состоит из 4 семантических сетей для основных знаменательных частей речи:
существительных, глаголов, прилагательных и наречий.

Этапы применения WordNet к модели, описанной в предыдущем разделе:
\begin{enumerate}[1)]
    \item провести грамматический разбор запроса, выявить части речи, соответствующие словам;
    \item определить (хотя бы приближенно) возможную семантику слова в соответствии с контекстом;
    \item найти с использованием WordNet похожие слова (согласно метрике семантической близости);
    \item ранжировать поисковую выдачу в соответствии с семантической близостью слов и $n$-грамм для
          последующего обучения нейронной сети (с помощью классических алгоритмов либо субъективно);
    \item обучить нейросеть и провести сравнение результатов.
\end{enumerate}

Семантическая близость $n$-грамм может быть определена как среднее геометрическое коэффициентов близости отдельных слов:
\begin{equation}
    \label{eq:ngram-sim}
    \begin{aligned}
        S(u_1u_2\dots u_n, w_1w_2\dots w_n) = \left( \prod\limits_{i=1}^n {S(u_i, w_i)} \right)^{\frac1n}= \\
        = \sqrt[n]{S(u_1, w_1)S(u_1, w_2)\dots S(u_n, w_n)},
    \end{aligned}
\end{equation}
где $u_1u_2\dots u_n$ и $w_1w_2\dots w_n$ --- сравниваемые $n$-граммы (здесь мы считаем, что порядок слов имеет значение).

Формула \eqref{eq:ngram-sim} может быть продиктована следующими соображениями:
\begin{enumerate}[1)]
    \item коэффициент близости должен равняться некоторому <<среднему>> из коэффициентов близости отдельных слов;
    \item с другой стороны, наличие неподобных пар слов должно приводить к значительному (но не полному) снижению
          близости всей $n$-граммы (что позволяет отсечь, например, среднее арифметическое).
\end{enumerate}

Одним из базовых методов для выведения семантики слова в контексте (разрешения неоднозначности, англ. disambiguation)
является метод Леска \cite{10.1145/318723.318728}, разработанный инженером компании Bell Labs М. Леском в 1986 году.
Идея метода заключается в поиске значения слова в списке словарных определений с учетом контекста, где это слово использовано.
Основным критерием для выбора значения послужило следующее правило: заложенный в этом определении смысл должен был частично
совпадать со смыслом значений соседних слов в контексте.

Алгоритм Леска работает в следующей последовательности:
\begin{enumerate}[1)]
    \item На первом шаге алгоритма отделяется контекст для рассматриваемого слова --- чаще всего, не более 10 расположенных
          рядом слов.
    \item Далее, для рассматриваемого слова (или его начальной формы) ищутся все определения в некоторой базе знаний
          (например, толковом словаре, хотя данный вариант редко используется непосредственно).
    \item Затем происходит поиск слов из контекста в каждом найденном определении. Если какое-либо слово из контекста
          присутствует в определении, то данное определение получает "<балл"> --- т. е. увеличивается счетчик совпадений.
    \item Наконец, происходит ранжирование определений по значениям счетчика совпадений. Чем выше значение счетчика, тем более
          подходящим к контексту считается определение.
\end{enumerate}

Рассмотрим следующий пример: согласно Большому толковому словарю русского языка С. А. Кузнецова \cite{kuznecov2008noveishiy},
слово "<штанга"> в русском языке имеет следующие значения:
\begin{enumerate}[1)]
    \item металлический стержень, используемый как деталь во многих механизмах;
    \item боковая стойка (иногда и верхняя перекладина) футбольных, хоккейных и т.п. ворот;
    \item снаряд для занятий тяжёлой атлетикой, состоящий из металлического стержня, на концах которого укреплены
          съёмные диски различного веса.
\end{enumerate}

Теперь рассмотрим следующие предложения, содержащие слово "<штанга">:
\begin{enumerate}[1)]
    \item Троллейбусная штанга, как правило, изготовляется из металлической трубы переменного сечения.
    \item В футбольном матче между "<Спартаком"> и "<Зенитом"> форвард ленинградцев дважды поразил штангу, а защитник москвичей
          отметился голом в свои ворота.
    \item На чемпионате мира по тяжелой атлетике наш спортсмен поднял штагу рекордного веса.
\end{enumerate}

Как видим, в первом предложении присутствует слово "<металлический">, которое есть только в первом определении слова "<штанга">.
Во втором предложении такую же роль играют слова "<футбольный"> и "<ворота">, в третьем --- "<вес"> и "<тяжелая атлетика">.
Таким образом, в данном примере алгоритм работает безупречно, правильно определяя значение слова в зависимости от контекста.

Чаще, к сожалению, возникает обратная ситуация --- а именно, когда словарное определение оказывается недостаточно емким и не включает
в себя наиболее часто встречающиеся в контексте слова. Поэтому чаще применяются модификации алгоритма, основанные не только на
словарных определениях, но и на примерах употребления слов в данных значениях в контексте.

Для применения алгоритма воспользуемся методом, содержащимся непосредственно в библиотеке NLTK. Он возвращает для заданного слова
в предложении наиболее подходящий (с точки зрения алгоритма) набор синонимов, иначе --- синонимический ряд (англ. synset),
представляющий из себя слова со схожим значением, объединенные в узел семантической сети. В WordNet каждый такой набор дополнен
определением и примерами употребления слов в контексте. Слова, имеющие несколько значений, включаются в несколько синонимических рядов
(выбор между которыми в контексте заданного поискового запроса или предложения будет осуществляться с помощью алгоритма Леска)
и могут быть причислены к различным синтаксическим и лексическим классам.

Синонимические ряды, в отличие от лишенных контекста слов, уже подлежат количественной оценке схожести.

В качестве метрики семантической близости воспользуемся метрикой Ву-Палмер \cite{10.3115/981732.981751}, предложенной в 1994 году
специалистами по компьютерной лингвистике Чжибяо Ву и Мартой Палмер. Она вычисляется по следующей формуле
\begin{equation}
    \label{eq:wup-sim}
    S_{WP} = 2\times\frac{d(\mathrm{LCS}(s_1, s_2))}{d(s_1) + d(s_2)}
\end{equation}
Здесь $d(s)$ --- глубина синонимического ряда $s$ в таксономии WordNet, $\mathrm{LCS}(s_1, s_2)$ --- наиболее специфический
(то есть, наименее общий) узел (англ. least common subsumer), являющийся предком как $s_1$, так и $s_2$.

Заметим, что метрика \eqref{eq:wup-sim} определена не всегда, поскольку таксономия может не быть деревом (в общем случае, это лес),
а значит, $\mathrm{LCS}(s_1, s_2)$ для произвольных $s_1$ и $s_2$ может не существовать. Однако же, в нашем случае, поскольку мы
во всех случаях вычисляем метрику подобия между схожими словами, имеющими общего предка в таксономии.

Если метрика \eqref{eq:wup-sim} определена, то всегда $0 < S_{WP}\leqslant 1$ (заметим, что она всегда строго больше 0, поскольку
глубина корня в таксономии по определению считается равной 1).

Для некоторых частей речи в WordNet (в частности, прилагательных и наречий) таксономия не организована в виде иерархии, поэтому
вычисление $S_{WP}$ становится невозможным. Наиболее простым подходом в таком случае является назначение таким парам синонимических
рядов константной метрики, то есть, фиксированного значения $0 < \alpha < 1$ (значение 1 используется только в случае, когда $s_1=s_2$,
то есть, слово считается максимально близко семантически самому себе). В нижеописанных экспериментах используется данный подход,
применяется значение $\alpha=0{,}2$.

Далее, нам необходимо модифицировать метрики TF-IDF \eqref{eq:norm-tf} и \eqref{eq:shifted-idf} для того, чтобы они учитывали вхождение
в тексты документов не только самих слов, но и их синонимов (включая семантическую близость). Наиболее естественным для этого представляется
подход с использованием взвешенного количества слов:

\begin{equation}
    \label{eq:adjusted-word-count}
    \tilde{N}_w(d) = \sum\limits_{s\in\mathrm{Syn}(w)} N_s(d) S_{WP}(w, s)
\end{equation}
Здесь $\tilde{N}_w(d)$ --- количество вхождений слова $w$ и его синонимов (с учетом близости) в документ $d$, $\mathrm{Syn}(w)$ --- множество
синонимов слова $w$ (подразумевается, что слово считается синонимом самого себя, т.е. $w\in\mathrm{Syn}(w)$ и $S_{WP}(w, w)=1$).

Заметим, что "<количество">, определенное по формуле \eqref{eq:adjusted-word-count}, может и не быть целым числом (что очевидно следует из
определения). Однако, это не влияет на корректность вычислений метрик TF-IDF по формулам \eqref{eq:norm-tf} и \eqref{eq:shifted-idf}.

\subsubsection{Источники текстовых документов и база данных}
В рассматриваемой модели использовались те же источники (см. стр. \pageref{lab:sources-grammar}) и та же база данных (стр. \pageref{lab:db-grammar}),
что и в модели на основе грамматической структуры текстов, рассмотренной в предыдущем разделе.

\subsubsection{Генеративно-состязательная сеть}
Помимо топологии генеративно-состязательной сети, использованной в предыдущей модели (см. стр. \pageref{lab:gan-grammar}), была использована модель с
с добавлением дополнительного слоя с 2048 нейронами. Структура всей сети выглядит следующим образом:
\begin{itemize}
    \item для подсети $G$:
          \begin{itemize}
              \item входной слой с размером 100 (для случайного
                    входа);
              \item слой с 256 нейронами, с использованием выпрямителя с протечкой при $\alpha = 0.2$;
              \item слой с 512 нейронами, с использованием выпрямителя с протечкой при $\alpha = 0.2$;
              \item слой с 1024 нейронами, с использованием выпрямителя с протечкой при $\alpha = 0.2$;
              \item слой с 2048 нейронами, с использованием выпрямителя с протечкой при $\alpha = 0.2$;
              \item выходной слой с размером 32, использующий гиперболический тангенс в качестве функции активации.
          \end{itemize}
    \item для подсети $D$:
          \begin{itemize}
              \item входной слой с размером 32 (для подачи векторов параметров, генерируемых подсетью G);
              \item слой с 2048 нейронами, использующий ReLU с $\alpha = 0.2$;
              \item слой отсева с коэффициентом 0,3;
              \item слой с 1024 нейронами, использующий ReLU с $\alpha = 0.2$;
              \item слой отсева с коэффициентом 0,3;
              \item слой с 512 нейронами, использующий ReLU с $\alpha = 0.2$;
              \item слой отсева с коэффициентом 0,3;
              \item слой с 256 нейронами, использующий ReLU с $\alpha = 0.2$;
              \item слой отсева с коэффициентом 0,3;
              \item выходной слой с размерностью 1 (скалярное значение, обозначающее оценку релевантности).
          \end{itemize}
\end{itemize}

\subsubsection{Полученные результаты}
Результаты для обеих конфигураций генеративно-состязательных сетей представлены в таблицах \ref{tab2} и \ref{tab3}, а также на рис. \ref{fig:wn-scores-1}
и  \ref{fig:wn-scores-2}.
\begin{table}[tbp]
    \caption{Результаты поисковых запросов в модели с 3 слоями}
    \begin{center}
        \begin{tabular}{ccc}
            \toprule
            \textbf{Запрос}                           & \multicolumn{2}{c}{\textbf{Оценка релевантности}}                               \\
                                                      & \textbf{\textit{Top 10}}                          & \textbf{\textit{Bottom 10}} \\
            \midrule
            india  fifteen minutes more he passed     & \(\mu=0.6021\)                                    & \(\mu=0.3059\)              \\
                                                      & \(\sigma=0.2263\)                                 & \(\sigma=0.0312\)           \\
            \midrule
            that he had also seen her                 & \(\mu=0.8491\)                                    & \(\mu=0.2677\)              \\
                                                      & \(\sigma=0.0158\)                                 & \(\sigma=0.3058\)           \\
            \midrule
            more serious than by you do know          & \(\mu=0.8759\)                                    & \(\mu=0.1647\)              \\
                                                      & \(\sigma=0.0506\)                                 & \(\sigma=0.0233\)           \\
            \midrule
            answer that he asked questions to come to & \(\mu=0.8948\)                                    & \(\mu=0.8529\)              \\
                                                      & \(\sigma=0.0080\)                                 & \(\sigma=0.0506\)           \\
            \midrule
            was a bull turned from sam                & \(\mu=0.8709\)                                    & \(\mu=0.5615\)              \\
                                                      & \(\sigma=0.0355\)                                 & \(\sigma=0.3926\)           \\
            \midrule
            likely place that was with such matters   & \(\mu=0.7994\)                                    & \(\mu=0.3423\)              \\
                                                      & \(\sigma=0.1651\)                                 & \(\sigma=0.3689\)           \\
            \midrule
            and when i tried to the greatest          & \(\mu=0.9092\)                                    & \(\mu=0.8777\)              \\
                                                      & \(\sigma=0.0236\)                                 & \(\sigma=0.0165\)           \\
            \midrule
            i knew una had wept  he                   & \(\mu=0.6551\)                                    & \(\mu=0.3681\)              \\
                                                      & \(\sigma=0.1944\)                                 & \(\sigma=0.0360\)           \\
            \bottomrule
        \end{tabular}\label{tab2}
    \end{center}
\end{table}
\begin{table}[tbp]
    \caption{Результаты поисковых запросов в модели с 4 слоями}
    \begin{center}
        \begin{tabular}{ccc}
            \toprule
            \textbf{Запрос} & \multicolumn{2}{c}{\textbf{Оценка релевантности}}                               \\
                            & \textbf{\textit{Top 10}}                          & \textbf{\textit{Bottom 10}} \\
            \midrule
            then true for its sake on himself         & \(\mu=0.6028\)                                    & \(\mu=0.0888\)              \\
                                                      & \(\sigma=0.3811\)                                 & \(\sigma=0.0567\)           \\
            \midrule
            she answered   presently i said           & \(\mu=0.8736\)                                    & \(\mu=0.0272\)              \\
                                                      & \(\sigma=0.0147\)                                 & \(\sigma=0.0028\)           \\
            \midrule
            within hollow her large eyes large full   & \(\mu=0.4586\)                                    & \(\mu=0.0408\)              \\
                                                      & \(\sigma=0.4197\)                                 & \(\sigma=0.0049\)           \\
            \midrule
            how you did  i have been                  & \(\mu=0.5236\)                                    & \(\mu=0.0110\)              \\
                                                      & \(\sigma=0.4211\)                                 & \(\sigma=0.0019\)           \\
            \midrule
            other but he found his courage at least   & \(\mu=0.8612\)                                    & \(\mu=0.6925\)              \\
                                                      & \(\sigma=0.0046\)                                 & \(\sigma=0.1937\)           \\
            \midrule
            my life very that was than ever           & \(\mu=0.7160\)                                    & \(\mu=0.2305\)              \\
                                                      & \(\sigma=0.3386\)                                 & \(\sigma=0.3608\)           \\
            \midrule
            deem  remember that he was interests      & \(\mu=0.7913\)                                    & \(\mu=0.3335\)              \\
                                                      & \(\sigma=0.1954\)                                 & \(\sigma=0.2972\)           \\
            \midrule
            oh don  your true taste white             & \(\mu=0.1604\)                                    & \(\mu=0.0718\)              \\
                                                      & \(\sigma=0.1300\)                                 & \(\sigma=0.0042\)           \\
            \bottomrule
        \end{tabular}\label{tab3}
    \end{center}
\end{table}

\begin{figure}
    \centerline{\includegraphics[scale=0.8]{312-1_scores.eps}}
    \caption{Оценки релевантности для модели с 3 слоями}\label{fig:wn-scores-1}
\end{figure}

\begin{figure}
    \centerline{\includegraphics[scale=0.8]{312-2_scores.eps}}
    \caption{Оценки релевантности для модели с 4 слоями}\label{fig:wn-scores-2}
\end{figure}

Заметим, что при использовании обеих топологий прослеживается четкая грань между псевдорелевантными и псевдонерелевантными результатами поисковой выдачи.
Следует также заметить, что сравнение абсолютных оценок релевантности для разных запросов смысла не имеет, важны лишь относительные оценки для каждого 
конкретного запроса в отдельности (поскольку при ранжировании результатов каждый запрос обрабатывается по отдельности, а выдача ранжируется (сортируется)
в соответствии со значением оценки, которе используется лишь для определения позиции (ранга) результата в выдаче по запросу).

На основании полученных результатов мы можем сразу же выявить преимущества и недостатки модели,
основанной на WordNet, по сравнению с простейшей моделью, учитывающей лишь грамматические словоформы:
\begin{itemize}
    \item увеличился охват запросов документами, в связи с тем, что, в отличие от предыдущей модели, в рассматриваемой
          учитываются не только сами слова, входящие в запрос, но и семантически близкие им. Это в некоторых случаях позволило
          увеличить точность запросов и релевантность результатов, что показало больший разброс в поисковой выдаче и дало
          возможность продемонстрировать различия в псевдорелевантных и псевдонерелевантных результатах;
    \item с другой стороны, контекст запроса часто не позволяет однозначно определить семантику того или иного слова 
          в типологии WordNet. В частности, в ходе подготовки данных для моделей и отладки написанных программ было
          выяснено, что в ходе применения алгоритма Леска для часто распространенных английских слов I (я) и as (союз "<как">)
          в качестве синонимов часто упоминались слова indium (индий) и arsenic (мышьяк). Это связано с символами I и As для
          соответствующих химических элементов --- индия и мышьяка, которые практически во всех случаях не имели реального
          отношения к запросу;
    \item WordNet не позволяет опредлить на уровне метрики подобие некоторых частей речи (в частности, прилагательных), но применение
          константного значения для пар синонимических рядов с неопределенной метрикой нельзя считать исчерпывающим решением проблемы,
          поскольку прилагательные также могут иметь неравноценные синонимы.
\end{itemize}
В целом, WordNet является базой знаний для текстов общего назначения, поэтому ее применение к специализированным корпусам текстов 
(специализация пространства поиска) не всегда может давать лучшие результаты (поэтому перспективным представляется использование модели,
описанной в следующем разделе, поскольку она может быть обучена на исходных данных того же характера, что и входящие в пространство поиска 
(включая сами данные из пространства поиска)). Подводя итог сказанному, можем заключить, что модель с использованием WordNet имеет как
свои преимущества, так и недостатки по сравнению с простейшей моделью, однако путем грамотного и обоснованного подбора баз знаний (в частности,
таксономий типа WordNet) и настройки параметров можно добиться улучшения результатов и использования всех преимуществ WordNet (следует, однако,
заметить, что процесс этот требудет значительного объема "<ручного"> (то есть, неавтоматизируемого, требующего серьезного вмешательства человека) труда,
что не во всех случаях является лучшим решением).
\printbibliography[title={Список использованных источников},category=cited]
\printbibliography[title={Непроцитированные источники (должно быть пусто)},notcategory=cited]
\end{document}