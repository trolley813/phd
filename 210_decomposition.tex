Проблема разработки методов оценки релевантности с использованием нейросетевого подхода может быть
разбита на подзадачи следующим образом:
\begin{enumerate}[1)]
    \item Создание базы поисковых запросов. Данная база является необходимой как для формирования эталонного
    набора релевантных (псевдорелевантных) запросов, требуемого для обучения нейронных сетей, так и для последующего
    тестирования и оценки качества получаемой модели.
    \item Анализ конфигураций нейронных сетей для модели оценки релевантности. В зависимости от выбранной модели,
    ее параметризации и т. д. могут потребоваться различные конфигурации и топологии нейросетей. 
    \item Проектирование эффективной системы хранения данных. Таковая может включать в себя как таблицы в реляционной
    БД, так и нереляционные механизмы хранения (пары <<ключ-значение>>, XML- либо JSON-файлы (документо-ориентированная БД))
    и т. п. Необходимо выявить наиболее подходящие и эффективные в плане быстродействия варианты.
    \item Реализация программной системы и ее модулей. Под модулями здесь понимаются компоненты программной системы,
    отвечающие за реализацию конкретной задачи в ней. К задачам относятся как основные, являющиеся частью исследования,
    так и вспомогательные, необходимые для работы системы в целом и возможности проверки в ней гипотез, выдвинутых в диссертации.
    \item Обучение нейронных сетей и получение результатов. В качестве таковых здесь понимаются суждения о релевантности
    поисковой выдачи для конкретных запросов.
    \item Сопоставление результатов, полученных обеими типами методик (предлагаемыми в данной работе и существующими). 
    Выводы о целесообразности выбранного подхода.
\end{enumerate}