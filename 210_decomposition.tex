Проблема разработки методов оценки релевантности с использованием нейросетевого подхода может быть
разбита на подзадачи следующим образом:
\begin{enumerate}[1)]
    \item Создание базы поисковых запросов. Данная база является необходимой как для формирования эталонного
          набора релевантных (псевдорелевантных) запросов, требуемого для обучения нейронных сетей, так и для последующего
          тестирования и оценки качества получаемой модели.
    \item Анализ конфигураций нейронных сетей для модели оценки релевантности. В зависимости от выбранной модели,
          ее параметризации и т. д. могут потребоваться различные конфигурации и топологии нейросетей.
    \item Проектирование эффективной системы хранения данных. Таковая может включать в себя как таблицы в реляционной
          БД, так и нереляционные механизмы хранения (пары <<ключ-значение>>, XML- либо JSON-файлы (документо-ориентированная БД))
          и т. п. Необходимо выявить наиболее подходящие и эффективные в плане быстродействия варианты.
    \item Реализация программной системы и ее модулей. Под модулями здесь понимаются компоненты программной системы,
          отвечающие за реализацию конкретной задачи в ней. К задачам относятся как основные, являющиеся частью исследования,
          так и вспомогательные, необходимые для работы системы в целом и возможности проверки в ней гипотез, выдвинутых в диссертации.
    \item Обучение нейронных сетей и получение результатов. В качестве таковых здесь понимаются суждения о релевантности
          поисковой выдачи для конкретных запросов.
    \item Сопоставление результатов, полученных обеими типами методик (предлагаемыми в данной работе и существующими).
          Выводы о целесообразности выбранного подхода.
\end{enumerate}

Опишем подробно каждую из подзадач.

Для обучения генеративно-состязательной сети, способной различать релевантные и нерелевантные результаты поисковой выдачи
(как и любой другой GAN-сети), необходима база знаний --- набор эталонных (релевантных) поисковых результатов. Данные результаты
используются сетью в процессе ее обучения, чтобы она могла генерировать результаты поиска, в той или иной мере схожие с эталоном.
Кроме того, любая нейронная сеть для оценки качества ее работы нуждается в проверке --- подаче на сеть данных, отличных от тех,
на которых она обучалась (так называемой тестовой выборки). В зависимости от результатов работы сети на данной выборке, как правило,
делается вывод о пригодности модели для решения рассматриваемой задачи.

Далее, необходим анализ конфигураций и топологий используемой генеративно-состязательной сети. Это возможно как на теоретическом
(подробное рассмотрение анализируемой архитектуры сети и выводы о ее применимости к входным и выходным данным), так и на практическом
(построение модели и анализ ее качества путем непосредственного тестирования) уровнях.

Немаловажную роль играет и эффективное хранение данных. Необходимо подобрать используемую структуру данных для обеспечения приемлемой
скорости доступа к ним (в первую очередь это касается поисковой базы, так как релевантность результатов по запросам (поисковой выдачи)
и их ранжирование определяется перебором (итерацией) по всем документам, входящим в базу, либо по существенному их подмножеству).
В рамках этого подбора находятся определение (как правило, экспериментальным путем) систем хранения данных и сопутствующего программного
обеспечения, а также его конфигурация, дающая оптимальные результаты для поставленной задачи (например, в плане скорости доступа).

Следующая (наиболее объемная в плане трудозатрат) подзадача --- реализация программной системы, решающей поставленную задачу. Под 
реализаций подразумевается:
\begin{itemize}
      \item подбор и анализ используемых программных средств (языков программирования, библиотек и так далее), 
            необходимых для решения поставленной задачи;
      \item разработка и проектирование архитектуры программной системы --- программных модулей и механизмов взаимосвязи между ними;
      \item написание программного кода, реализующего необходимые алгоритмы для каждого из модулей, а также интеграция модулей между собой.
      \item тестирование и отладка написанных программ.
\end{itemize}

Далее идет непосредственное обучение спроектированных нейронных сетей, а также получение генерируемых ими результатов. Данная подзадача
представляется наиболее объемной в плане использованного машинного времени (CPU time), однако наименее трудозатратной в плане
работы исследователя (как творческой, так и "<рутинной">).

Last but not least\footnote{англ. "<последнее, но не наименьшее (по важности)">; фразеологизм труднопереводим на русский язык}
проводится сопоставление результатов, полученных всеми рассматриваемыми и исследованными моделями. Данное сопоставление проводится
как на объективном (численном, путем интерпретации результатов), так и на субъективном (поскольку речь идет о релевантности,
имеющей в первую очередь субъективное значение) уровнях. На основании результатов сопоставления делаются выводы о применимости
исследованных моделей к решению поставленной задачи, качестве этих моделей, а также о том, какую из них следует применять в зависимости
от специфики поставленной задачи (ее конкретными обстоятельствами).

Перейдем к методам и алгоритмам решения каждой из подзадач в отдельности.