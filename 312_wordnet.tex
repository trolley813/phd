
WordNet \cite{Miller95wordnet:a} --- лексическая база данных английского языка, разработанная в Принстонском университете. 
Она представляет собой электронный словарь-тезаурус и набор семантических сетей для английского языка. 

Словарь, представленный в WordNet, состоит из 4 семантических сетей для основных знаменательных частей речи: 
существительных, глаголов, прилагательных и наречий.

Этапы применения WordNet к модели, описанной в предыдущем разделе:
\begin{enumerate}[1)]
    \item провести грамматический разбор запроса, выявить части речи, соответствующие словам;
    \item определить (хотя бы приближенно) возможную семантику слова в соответствии с контекстом;
    \item найти с использованием WordNet похожие слова (согласно метрике семантической близости);
    \item 
\end{enumerate}

Семантическая близость $n$-грамм может быть определена как среднее геометрическое коэффициентов близости отдельных слов:
\begin{equation}
    \label{eq:ngram-sim}
    \begin{aligned}
    S(u_1u_2\dots u_n, w_1w_2\dots w_n) = \left( \prod\limits_{i=1}^n {S(u_i, w_i)} \right)^{\frac1n}= \\
    = \sqrt[n]{S(u_1, w_1)S(u_1, w_2)\dots S(u_n, w_n)},
    \end{aligned}
\end{equation}
где $u_1u_2\dots u_n$ и $w_1w_2\dots w_n$ --- сравниваемые $n$-граммы (здесь мы считаем, что порядок слов имеет значение).

Формула \eqref{eq:ngram-sim} может быть продиктована следующими соображениями:
\begin{enumerate}[1)]
    \item коэффициент близости должен равняться некоторому <<среднему>> из коэффициентов близости отдельных слов;
    \item с другой стороны, наличие неподобных пар слов должно приводить к значительному (но не полному) снижению 
    близости всей $n$-граммы (что позволяет отсечь, например, среднее арифметическое).
\end{enumerate}