После того, как методы решения отдельных подзадач были определены, мы можем сформулировать метод решения всей проблемы в целом.
Основным, стержневым элементом в нем является использование подхода с использованием генеративно-состязательных нейронных сетей
в проблеме оценки релевантности результатов семантического поиска.

Первым этапом решения проблемы является проектирование архитектуры программной системы для решения поставленной задачи,
преимущественно "<крупноблочно">, в формате "<сверху вниз"> (иначе --- разбиение системы на модули, определение механизмов
взаимодействия модулей между собой, и только затем --- реализация алгоритмов в каждом из модулей по отдельности), как с 
теоретической, так и экспериментальной точек зрения.

Далее, когда блоки-модули реализованы, следует второй этап решения проблемы (преимущественно экспериментальный): обучение
нейронных сетей, анализ получаемых результатов и, в случае необходимости, корректировка выбранной параметризации (вполне
может оказаться, что изначально выбранные значения параметров либо дают неудовлетворительные результаты, либо приводят
к ошибкам в ходе обучения, делая невозможным дальнейший процесс).

Следующий шаг в решении проблемы --- систематизация полученных результатов и их сопоставление с таковыми для существующих
методов решения проблемы. Данный этап также носит преимущественно экспериментальный характер (статистический анализ
результатов, их сравнение и тому подобное), однако может опираться и на некоторые теоретические идеи и рассуждения.

Финальным этапом в решении проблемы являются выводы о полученных результатах и о целесообразности подхода,
предложенного в диссертационной работе, к поставленной проблеме.