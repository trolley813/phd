Как следует из обзора предшествующих работ исследователей по проблеме
и предлагаемых ими путей ее решения, вопрос использования нейронных (в первую очередь,
генеративно-состязательных) сетей для оценки релевантности поисковых запросов проработан
недостаточно глубоко. По итогам анализа примерно двух десятков статей, близких к
теме исследования (а именно, брались все статьи по теме использования нейронных сетей
применительно к задачам информационного поиска) было установлено, что лишь одна статья
упоминает генеративно-состязательные сети в контексте их применения в задачах ранжирования
результатов поиска. Более того, в данной статье рассмотрена лишь общая возможность такого
применения, без каких-либо конкретных примеров и результатов.

Таким образом, можно сделать выводы о том, что исследование вопроса применимости генеративно-состязательных
сетей в качестве механизма оценки релевантности результатов поиска будет составлять научную новизну
и представлять академический интерес. Кроме того, данная задача представляет и интерес практический,
поскольку основная часть создаваемой и обрабатываемой информации в настоящее время запрашивается
в использованием информационно-поисковых систем, то есть, разработка таких методов потенциально
может содействовать улучшению качества поисковой выдачи.

Резюмируя вышесказанное, автор делает выводы о необходимости проведения собственных исследований
по теме диссертации.