В связи со всем вышеизложенным, можно констатировать, что подход с использованием нейросетей
(в частности, генеративно-состязательных) на настоящий момент довольно слабо проработан,
и исследования в этой области могут представлять научную новизну. Определим основные
задачи, составляющие предмет исследования данной работы:
\begin{enumerate}[1)]
    \item анализ и разработка подхода к проблеме с использованием нейросетевых технологий;
    \item разработка архитектуры программной системы и исходных данных для проверки
          выдвигаемых гипотез;
    \item подбор наборов данных;
    \item подбор алгоритмов и методов для решения промежуточных задач, не связанных непосредственно
          с исследуемой проблемой;
    \item выбор конфигураций и топологий нейронных сетей для решения поставленных задач;
    \item обучение сетей и их тестирование на выбранных наборах данных;
    \item анализ полученных результатов;
    \item сравнение результатов, полученных нейросетевыми методами, с таковыми для
          существующих классических методов;
    \item выводы о целесообразности применения нейросетевого подхода к поставленной проблеме.
\end{enumerate}
Разберем каждую задачу по отдельности.

В первом пункте проводится анализ поставленной проблемы в целом и ее разбиение на части, после
чего выявляются подзадачи, требующие нейросетевых алгоритмов для своего решения.

Далее, с учетом ранее проведенного анализа, разрабатывается архитектура используемой для
тестирования выбранного подхода и последующего подтверждения (опровержения) гипотез программной системы
(в данном случае --- информационно-поисковой). Каждый модуль системы проектируется по отдельности,
после чего они соединяются в целостную систему.

Далее следует подбор необходимых наборов данных, на которых планируется тестировать выдвигаемые гипотезы
в разработанной информационно-поисковой системе. Необходимо, чтобы исходные данные позволяли в полной
мере продемонстрировать результаты и, как следствие, сделать выводы о целесообразности предлагаемого
в диссертации подхода.

В следующем пункте проводится реализация тех модулей и компонентов программной системы, которые в рамках
решения поставленных задач не требуют применения нейронных сетей. Данные модули, как правило, реализованы
с использованием готовых программных решений и библиотек (применимость и пригодность таковых, их соответствие
целям исследования также является подзадачей, требующей анализа).

Затем следует этап, имеющий наибольшую научную ценность и составляющий значительную часть научной новизны,
содержащейся в диссертации, --- а именно, подбор конфигураций нейронных сетей, пригодных для решения
задачи. Исследование и анализ проходит с учетом результатов, полученных при проверке каждой из исследуемых
конфигураций.

Далее следует технический процесс обучения каждой из предлагаемых моделей и проверка их на тестовых выборках
из подобранных ранее наборов данных.

Следующим этапом проводится анализ полученных результатов, подбор метрик, позволяющих проиллюстрировать,
насколько эти результаты хороши или плохи, для того, чтобы сделать соответствующие выводы.

Затем полученные результаты сравниваются с таковыми у существующих методов (преимущественно реализованных
с помощью классической алгоритмизации), как субъективным путем (например, сравнение содержательной
релевантности запросов), так и по каким-либо метрикам, позволяющим формализовать это сравнение, выразить
его в численном виде.

Наконец, на основании всего вышеизложенного, делаются выводы о возможности применения нейросетевых алгоритмов
к задаче оценки релевантности, также о моделях реализации, конфигурациях и топологиях нейронных сетей,
дающих лучшие результаты в решении поставленной задачи.

Перечисленные выше этапы составляют предмет исследования, проведенного в данной работе.