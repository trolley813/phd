%!TEX encoding = UTF-8

Ниже приводится обзор статей на тему использования методов машинного обучения в системах информационного поиска и, в частности,
их применения к оценке релевантности, опубликованных за последнее время (2016--2020 годы).

В статье \cite{DBLP:journals/corr/MitraC17} описываются различные архитектуры сетей глубинного обучения применительно к задачам
информационного поиска. Авторы статьи затрагивают в исследовании задачи как распознавания поискового запроса, так и оценки
релевантности результатов. Однако, для последних описывается преимущественно подход, основанный на обучении "<с учителем">, когда
метки релевантности/нерелевантности запросов заданы заранее (ground truth). Генеративно-состя\-зательные сети в данной работе
не рассматриваются.

Статья \cite{DBLP:journals/corr/abs-1802-10078} описывает нейросетевую архитектуру поисковой системы, предназначенной для
текстов узкой направленности (научных статей в области биологии и медицины). В работе описана так называемая дельта-модель,
состоящая из сверточной подсети, за которой следует подсеть с прямой связью (feed-forward). Сверточная модель использует метрики
схожести (similarity), основанные на учете $n$-грамм.

В качестве обучающей выборки в \cite{DBLP:journals/corr/abs-1802-10078} используются журналы кликов (click logs) поисковых систем,
которые в той или иной мере позволяют судить о релевантности полученных пользователями результатов, то есть, являются своего рода
метками релевантности. Таким образом, можно говорить о схожести методов, предлагаемых в \cite{DBLP:journals/corr/MitraC17} и
\cite{DBLP:journals/corr/abs-1802-10078}.

Статья \cite{DBLP:journals/corr/abs-2001-09896} посвящена семантической оценке важности слов в контексте документа (вариации
TF-IDF). На данный момент (28.02.2020) статья не завершена, однако в ней присутствуют необходимые результаты и их оценка, которая
заключается в сравнении семантической вариации с "<классической">. Несмотря на то, что авторы применяют полностью алгоритмические
методы, без использования нейросетей, статья представляет интерес для темы работы.

Публикация \cite{DBLP:journals/corr/abs-2001-07075} рассматривает проблему оценки релевантности в информационном поиске несколько
с другой стороны --- использования квантовоподобной (quantum-like) структуры для суждений о релевантности. Авторы сравнивают
предложенную ими квантово-вероятностную модель с байесовской. Опять же, нейросетевые алгоритмы авторами не рассматриваются,
но, тем не менее, данный метод также представляет существенный интерес.

В статье \cite{DBLP:journals/corr/abs-1910-00314} рассматриваются задачи оценки релевантности запросов применительно к узкой
предметной области --- биологии, медицины и здравоохранения. В качестве документов авторы рассматривают аннотации к публикациям,
размещенным в системе PubMed. В публикации описываются различные способы представления запросов, документов и предложений, включая
и TF-IDF (аналогично \cite{DBLP:journals/corr/abs-2001-09896}). Авторы выделяют 2 основные задачи --- нейросетевой модели для
ранжирования (с использованием SVM для первичного ранжирования и "<классических"> функций по типу BM25 для повторного), а также
построения моделей, использующих множество представлений (multi-view), таких, как TF-IDF и модель "<мешка слов"> (bag-of-words).
GAN-сети в данной работе не рассматриваются, однако, во второй поставленной авторами задаче используются в том числе и методы
обучения без учителя.

Имеет смысл упомянуть и работу \cite{DBLP:journals/corr/abs-1909-06859}, в которой предлагается модель MarlRank многоагентного
обучения ранжированию с подкреплением (конкретные архитектуры нейросетей авторами не описываются). В данной работе авторы
рассматривают каждый документ как агент в марковском процессе принятия решений. Предсказание релевантности документом
осуществляется на основе как его собственных характеристик, так и характеристик схожих документов, в связи с чем (оценка
релевантности на основе конечного набора характеристик) публикация также включена в данный обзор как представляющая интерес.

Статья \cite{DBLP:journals/corr/abs-1907-08657} поднимает проблему ограниченной доступности меток релевантности запросов,
оцениваемых непосредственно пользователями-экспертами (а эта проблема является довольно актуальной для поставленных задач).
Авторами предлагаются методы повышения такой доступности путем генерации подобных меток и их отбора с использованием в
том числе нейросетевых алгоритмов, в связи с чем данная работа также довольно близка поставленным задачам. Опять-таки,
генеративно-состя\-зательные сети в публикации не упомянуты.

Работа \cite{DBLP:journals/corr/abs-1908-06132} представляет собой кандидатскую (PhD) диссертацию соискателя, представляющего
один из университетов штата Нью-Йорк (США). Автор подробно рассматривает различные нейросетевые модели и их применение в информационном
поиске. Тема генеративно-состязательных сетей, равно как и тема нейросетевой оценки релевантности, не затрагивается.

Работа \cite{DBLP:journals/corr/abs-1906-09404} предлагает нейросетевую архитектуру RLTM(Reinforced Long-Text Matching), предназначенную для эффективного
ранжирования запросов к «длинным» документам. Хотя упоминаемые в работе архитектуры сетей не являются генеративно-состязательными
по своему принципу, в них также применяется принцип различения (дискриминации) между положительными и отрицательными результатами
(основанный на ненейросетевом подходе). Авторы также предлагают подход по отбору из текстов документов наиболее значимых
предложений, указывая критерии такой значимости.

Предметная область в публикации не указана, в качестве используемых наборов данных используются результаты поисковых запросов к
системам общего назначения (на китайском языке).

В публикации \cite{DBLP:journals/corr/abs-1904-06808} предлагается аксиоматический подход к регуляризации нейросетевых моделей
ранжирования. Хотя данная работа не относится напрямую к теме исследования, тем не менее, описанные в ней принципы и подходы
могут оказаться полезными при разработке нейросетевых архитектур применительно к поставленным задачам.

Статья \cite{DBLP:journals/corr/abs-1903-06902} представляет собой достаточно объемный обзор нейросетевых моделей оценки
релевантности и ранжирования результатов. Авторами рассматриваются как архитектуры моделей ранжирования, так и эмпирические
их оценки. Тем не менее, GAN-сети в работе не упоминаются, что может свидетельствовать о существенной научной новизне
исследования этой области.

Публикация \cite{DBLP:journals/corr/abs-1812-00073} посвящена библиотеке TF-Ranking, представляющей собой дополнение для
популярного набора TensorFlow, которое предназначено для обучения ранжированию. Несмотря на то, что статья является
по содержанию преимущественно технической, она представляет интерес в плане реализации подобных нейросетевых архитектур
в качестве дополнений к существующим библиотекам машинного обучения.

Также интерес представляет статья \cite{DBLP:journals/corr/abs-1810-12936}, которая предлагает нейросетевые модели
для подхода псевдообратной связи по релевантности (pseudo relevance feedback). Авторы отмечают сложности, возникающие при
комбинировании PRF с нейросетевыми моделями и в связи с этим предлагают единый каркас, объединяющий данные подходы.

В работе \cite{DBLP:journals/corr/abs-1809-01682} описывается глубинное ранжирование по релевантности (deep relevance ranking).
Авторы подробно рассматривают ряд архитектур на основе модели DRMM, их применимость к задачам ранжирования, а также проводят
тестирование созданных моделей на наборе данных BIOASQ для автоматизированных ответов на вопросы (question answering).
Данная тема является смежной к теме оценки релевантности поисковых запросов (если рассматривать заданный пользователем
вопрос в качестве поискового запроса, а ответ в качестве результата).

Публикация \cite{DBLP:journals/corr/abs-1807-05355} напрямую не относится к теме исследования, но, тем не менее, поднимает
любопытный вопрос — влияние психологического эффекта порядка (в котором пользователю представлены результаты поиска)
на многомерные системы суждения о релевантности на основе журналов запросов (query logs).

Значительный интерес для поставленной темы представляет статья \cite{DBLP:journals/corr/abs-1806-03577}. Данная публикация
является единственной, где поднимается тема генеративно-состязательных сетей применительно к задачам информационного поиска.
Основной рассматриваемой задачей является генерация поисковых запросов, методы же оценки релевантности с помощью GAN-сетей
в работе не рассмотрены, в качестве модели автор предлагает вышеописанную PRF (как вариант, основанную на нейросетевой
архитектуре).

Несмотря на то, что публикация \cite{DBLP:journals/corr/abs-1806-03577} носит преимущественно обзорный характер, не имея
практически никаких технических подробностей (однако, в ней приведены ссылки на используемое и упоминаемое в тексте
программное обеспечение), ее следует считать одним из основных источников при работе над темой исследования.

Также к данной теме относится работа \cite{DBLP:journals/corr/abs-1805-02184}, в которой рассматривается моделирование
многомерной релевантности с использованием векторных пространств. В контексте поставленных задач, статью можно отнести
к смежным, поскольку в ней рассматриваются исключительно алгоритмические (ненейросетевые) методы решения, и основная тема
статьи почти не связана с оценкой самой релевантности, однако данный материал следует считать ценным для исследования.

Публикация \cite{DBLP:journals/corr/abs-1711-08611} рассматривает DRMM для поиска по произвольному запросу (ad-hoc retrieval)

В статье \cite{DBLP:journals/corr/abs-1710-05649} предлагается модель DeepRank для глубинного обучения ранжированию,
симулирующая процесс оценки релевантности человеком. Архитектурно, она состоит из модуля, извлекающего контексты для оценки
релевантности, нейросети (сверточной, либо рекуррентной) для определения «локальных» релевантностей, а также агрегирующей
сети для вычисления «глобальной» оценки (по всему документу в целом). Согласно исследованию авторов, метод показывает высокие
результаты, сравнимые с текущими моделями глубинного обучения и обучения ранжированию, а иногда и превосходящие их.
Генеративно-состязательные сети в работе непосредственно не упоминаются, однако, для темы исследования данная публикация
также представляет интерес.

В работе \cite{DBLP:journals/corr/abs-1709-01709} затрагивается тема отбора результатов (семплинга, англ. sampling)
для широкомасштабной оценки поисковых результатов. Авторы предлагают ряд различных методов для создания коллекций запросов
(такие коллекции в большей степени требуют вмешательства человека, так как они используются для формирования меток
ground truth). Данную публикацию также стоит считать ценным материалом для темы исследования, в связи с тем,
что определенная часть оценки результатов будет проводиться субъективно.