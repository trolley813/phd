Функции оценки релевантности класса BM25 \cite{Amati2009} и аналогичные --- BM11, BM15 и др. \cite{bm25xapian}, а также модификации - BM25F, BM25+
\cite{10.1145/2063576.2063584, 10.1561/1500000019, Zaragoza04microsoftcambridge} по состоянию на 2021 год по-прежнему широко используются, по
крайней мере, в качестве эталонных классических алгоритмов оценки релевантности поисковой выдачи для последующего ранжирования. В соответствии
с этим мы можем утверждать, что полученные в предыдущих параграфах результаты одновременно служат и сопоставлением их с исследованиями 
предшественников, поскольку во всех моделях генеративно-состязательные сети обучались на псевдорелевантных (условно релевантных) результатах,
полученных с помощью ранжирования по значениям функции BM25 в одной из вариаций.

Результаты, полученные нейросетевыми моделями, демонстрируют приемлемое качество результатов в сравнении с таковыми, которые получаются при
ранжировании классическими алгоритмами. Вероятность ошибок (как первого, так и второго рода) невелика при условии корректного составления модели
(однако, как было показано выше, сравнительно качественные результаты получаются уже при использовании простых генеративно-состязательных
моделей с персептронной топологией, которые требуют небольшого объема ресурсов и машинного времени для их обучения). Таким образом, мы можем 
утверждать, что предложенный в диссертационной работе новый подход способен во всех рассмотренных случаях давать корректные суждения о релевантности
поисковых запросов с большой вероятностью.

Однако же, неисследованным остался вопрос о применении в качестве заведомо релевантной выборки результатов, отобранных вручную. Прежде всего,
это связано с большим объемом необходимых данных для подготовки, иными словами, это практически неосуществимо силами одного лишь автора.
Но, тем не менее, нет оснований полагать, что их использование приведет к значительному снижению качества результатов, вплоть до принципиальной
невозможности использования предложенного подхода на концептуальном уровне. Безусловно, не исключено, что для обучения моделей на подобных данных 
может понадобиться пересмотр параметризации и выбор в качестве параметров, подаваемых на вход нейросети, каких-либо иных характеристик результата
поискового запроса. Данная проблема может рассматриваться в качестве вопроса для последующей разработки темы исследования.

Кроме того, в работе была рассмотрена по большому счету лишь одна нейросетевая архитектура для построения генеративно-состязательных сетей.
Безусловно, в диссертации было убедительно показано, что и она уже дает приемлемые результаты как в плане их качества, так и характеристик
быстродействия сетевых моделей, однако же, неизвестно, является ли данная архитектура лучшей в вышеприведенных метриках. Это также представляет
практический и научный интерес для продолжения исследования по теме.