\documentclass{article}
\usepackage{tempora}
\usepackage[russian]{babel}
\usepackage[utf8]{inputenc}
\usepackage{csquotes}
\usepackage[bibstyle=gost-numeric,citestyle=gost-numeric]{biblatex}
\usepackage[margin=18mm]{geometry}
\usepackage[14pt]{extsizes}
\usepackage[bigdelims,vvarbb]{newtxmath}
\renewcommand{\baselinestretch}{1.5}
\bibliography{01_review}
\begin{document}
Ниже приводится обзор статей на тему использования методов машинного обучения в системах информационного поиска и, в частности, 
их применения к оценке релевантности, опубликованных за последнее время (2016--2020 годы).

В статье \cite{DBLP:journals/corr/MitraC17} описываются различные архитектуры сетей глубинного обучения применительно к задачам
информационного поиска. Авторы статьи затрагивают в исследовании задачи как распознавания поискового запроса, так и оценки 
релевантности результатов. Однако, для последних описывается преимущественно подход, основанный на обучении "<с учителем">, когда
метки релевантности/нерелевантности запросов заданы заранее (ground truth). Генеративно-состя\-зательные сети в данной работе
не рассматриваются.

Статья \cite{DBLP:journals/corr/abs-1802-10078} описывает нейросетевую архитектуру поисковой системы, предназначенной для
текстов узкой направленности (научных статей в области биологии и медицины). В работе описана так называемая дельта-модель, 
состоящая из сверточной подсети, за которой следует подсеть с прямой связью (feed-forward). Сверточная модель использует метрики
схожести (similarity), основанные на учете $n$-грамм.

В качестве обучающей выборки в \cite{DBLP:journals/corr/abs-1802-10078} используются журналы кликов (click logs) поисковых систем, 
которые в той или иной мере позволяют судить о релевантности полученных пользователями результатов, то есть, являются своего рода
метками релевантности. Таким образом, можно говорить о схожести методов, предлагаемых в \cite{DBLP:journals/corr/MitraC17} и 
\cite{DBLP:journals/corr/abs-1802-10078}.

Статья \cite{DBLP:journals/corr/abs-2001-09896} посвящена семантической оценке важности слов в контексте документа (вариации 
TF-IDF). На данный момент (28.02.2020) статья не завершена, однако в ней присутствуют необходимые результаты и их оценка, которая
заключается в сравнении семантической вариации с "<классической">. Несмотря на то, что авторы применяют полностью алгоритмические
методы, без использования нейросетей, статья представляет интерес для темы работы.

Публикация \cite{DBLP:journals/corr/abs-2001-07075} рассматривает проблему оценки релевантности в информационном поиске несколько
с другой стороны --- использования квантовоподобной (quantum-like) структуры для суждений о релевантности. Авторы сравнивают
предложенную ими квантово-вероятностную модель с байесовской. Опять же, нейросетевые алгоритмы авторами не рассматриваются,
но, тем не менее, данный метод также представляет существенный интерес.
 
В статье \cite{DBLP:journals/corr/abs-1910-00314} рассматриваются задачи оценки релевантности запросов применительно к узкой
предметной области --- биологии, медицины и здравоохранения. В качестве документов авторы рассматривают аннотации к публикациям,
размещенным в системе PubMed. В публикации описываются различные способы представления запросов, документов и предложений, включая
и TF-IDF (аналогично \cite{DBLP:journals/corr/abs-2001-09896}). Авторы выделяют 2 основные задачи --- нейросетевой модели для
ранжирования (с использованием SVM для первичного ранжирования и "<классических"> функций по типу BM25 для повторного), а также
построения моделей, использующих множество представлений (multi-view), таких, как TF-IDF и модель "<мешка слов"> (bag-of-words).
GAN-сети в данной работе не рассматриваются, однако, во второй поставленной авторами задаче используются в том числе и методы
обучения без учителя.

Имеет смысл упомянуть и работу \cite{DBLP:journals/corr/abs-1909-06859}, в которой предлагается модель MarlRank многоагентного
обучения ранжированию с подкреплением (конкретные архитектуры нейросетей авторами не описываются). В данной работе авторы 
рассматривают каждый документ как агент в марковском процессе принятия решений. Предсказание релевантности документом 
осуществляется на основе как его собственных характеристик, так и характеристик схожих документов, в связи с чем (оценка 
релевантности на основе конечного набора характеристик) публикация также включена в данный обзор как представляющая интерес. 

Статья \cite{DBLP:journals/corr/abs-1907-08657} поднимает проблему ограниченной доступности меток релевантности запросов,
оцениваемых непосредственно пользователями-экспертами (а эта проблема является довольно актуальной для поставленных задач).
Авторами предлагаются методы повышения такой доступности путем генерации подобных меток и их отбора с использованием в 
том числе нейросетевых алгоритмов, в связи с чем данная работа также довольно близка поставленным задачам. Опять-таки, 
генеративно-состя\-зательные сети в публикации не упомянуты.

Работа \cite{DBLP:journals/corr/abs-1908-06132} представляет собой кандидатскую (PhD) диссертацию соискателя, представляющего
один из университетов штата Нью-Йорк (США)...


\printbibliography
\end{document}