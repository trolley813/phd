\documentclass{article}
\usepackage{tempora}
\usepackage[russian]{babel}
\usepackage[utf8]{inputenc}
\usepackage{csquotes}
\usepackage[bibstyle=gost-numeric,citestyle=gost-numeric]{biblatex}
\usepackage[margin=18mm]{geometry}
\usepackage[14pt]{extsizes}
\usepackage[bigdelims,vvarbb]{newtxmath}
\renewcommand{\baselinestretch}{1.5}
\bibliography{01_review}
\begin{document}
Ниже приводится обзор статей на тему использования методов машинного обучения в системах информационного поиска и, в частности, 
их применения к оценке релевантности, опубликованных за последнее время (2016--2020 годы).

В статье \cite{DBLP:journals/corr/MitraC17} описываются различные архитектуры сетей глубинного обучения применительно к задачам
информационного поиска. Авторы статьи затрагивают в исследовании задачи как распознавания поискового запроса, так и оценки 
релевантности результатов. Однако, для последних описывается преимущественно подход, основанный на обучении "<с учителем">, когда
метки релевантности/нерелевантности запросов заданы заранее (ground truth). Генеративно-состя\-зательные сети в данной работе
не рассматриваются.

Статья \cite{DBLP:journals/corr/abs-1802-10078} описывает нейросетевую архитектуру поисковой системы, предназначенной для
текстов узкой направленности (научных статей в области биологии и медицины). В работе описана так называемая дельта-модель, 
состоящая из сверточной подсети, за которой следует подсеть с прямой связью (feed-forward). Сверточная модель использует метрики
схожести (similarity), основанные на учете $n$-грамм.

В качестве обучающей выборки в \cite{DBLP:journals/corr/abs-1802-10078} используются журналы кликов (click logs) поисковых систем, 
которые в той или иной мере позволяют судить о релевантности полученных пользователями результатов, то есть, являются своего рода
метками релевантности. Таким образом, можно говорить о схожести методов, предлагаемых в \cite{DBLP:journals/corr/MitraC17} и 
\cite{DBLP:journals/corr/abs-1802-10078}.

\printbibliography
\end{document}